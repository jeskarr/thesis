% \omiss produces '[...]'
\newcommand{\omissis}{[\dots\negthinspace]}

% Itemize symbols
% see: https://tex.stackexchange.com/a/62497
% \renewcommand{\labelitemi}{$\bullet$}
% \renewcommand{\labelitemii}{$\cdot$}
% \renewcommand{\labelitemiii}{$\diamond$}
% \renewcommand{\labelitemiv}{$\ast$}


\let\Chaptermark\chaptermark
% \Chaptername gives current chapter name
\def\chaptermark#1{\def\Chaptername{#1}\Chaptermark{#1}}
\makeatletter
% \currentname gives the current section name
\newcommand*{\currentname}{\@currentlabelname}
\makeatother

% Uncomment the following line for a different header/footer style
% \pagestyle{fancy} \setlength{\headheight}{14.5pt}
\fancyhead[L]{\fontsize{12}{14.5} \selectfont \thechapter. \Chaptername}
\fancyhead[R]{\fontsize{12}{14.5} \selectfont \currentname}
% Page number always in footer
\cfoot{\thepage}


% Custom hyphenation rules
\hyphenation {
    e-sem-pio
    ex-am-ple
}

% Images path, not using \graphicspath because it doesn't properly work with
% latexmk custom dependencies
\NewCommandCopy{\latexincludegraphics}{\includegraphics}
\renewcommand{\includegraphics}[2][]{\latexincludegraphics[#1]{../images/#2}}

% Page format settings
% see: http://wwwcdf.pd.infn.it/AppuntiLinux/a2547.htm
\setlength{\parindent}{14pt}    % first row indentation
\setlength{\parskip}{0pt}       % paragraphs spacing


% Load variables
\newcommand{\myName}{Jessica Carretta}
\newcommand{\myID}{2034312}
\newcommand{\myTitle}{Data visualization e infografiche: studio e applicazione attraverso la libreria D3.js}
\newcommand{\myDegree}{Tesi di laurea triennale}
\newcommand{\myUni}{Università degli Studi di Padova}
% For BSc level just use "Corso di Laurea" and don't add "Triennale" to it
\newcommand{\myFaculty}{Corso di Laurea in Informatica}
\newcommand{\myDepartment}{Dipartimento di Matematica ``Tullio Levi-Civita''}
\newcommand{\profTitle}{Prof.}
\newcommand{\myProf}{Lamberto Ballan}
\newcommand{\myLocation}{Longare}
\newcommand{\myAA}{2023-2024}
\newcommand{\myTime}{Settembre 2024}

% PDF file metadata fields
% when updating them delete the build directory, otherwise they won't change
\begin{filecontents*}{\jobname.xmpdata}
  \Title{Document's title}
  \Author{Author's name}
  \Language{it-IT}
  \Subject{Short description}
  \Keywords{keyword1\sep keyword2\sep keyword3}
\end{filecontents*}


% Acronyms
\newacronym[description={\glslink{llmg}{Large Language Model}}]
    {llm}{LLM}{Large Language Model}

\newacronym[description={\glslink{ictg}{Information and Communication Technology}}]
    {ict}{ICT}{Information and Communication Technology}

\newacronym[description={\glslink{erpg}{Enterprise Resource Planning}}]
    {erp}{ERP}{Enterprise Resource Planning}

\newacronym[description={\glslink{htmlg}{Hypertext Markup Language}}]
    {html}{HTML}{Hypertext Markup Language}

\newacronym[description={\glslink{cssg}{Cascading Style Sheets}}]
    {css}{CSS}{Cascading Style Sheets}

\newacronym[description={\glslink{jsg}{JavaScript}}]
    {js}{JS}{JavaScript}

\newacronym[description={\glslink{iclg}{In-Context Learning}}]
    {icl}{ICL}{In-Context Learning}

\newacronym[description={\glslink{fslg}{Few-Shot Learning}}]
    {fsl}{FSL}{Few-Shot Learning}

\newacronym[description={\glslink{vifg}{Visual Information Flow}}]
    {vif}{VIF}{Visual Information Flow}

\newacronym[description={\glslink{mbtig}{Myers-Briggs Type Indicator}}]
    {mbti}{MBTI}{Myers-Briggs Type Indicator}
    
\newacronym[description={\glslink{wcagg}{Web Content Accessibility Guidelines}}]
    {wcag}{WCAG}{Web Content Accessibility Guidelines}
    
\newacronym[description={\glslink{svgg}{Scalable Vector Graphics}}]
    {svg}{SVG}{Scalable Vector Graphics}
    
% Glossary entries
\newglossaryentry{datavizg} {
    name=Data visualization,
    text=Data Visualization,
    sort=datavisualization,
    description={È il processo di rappresentazione visiva dei dati attraverso grafici, tabelle e altre forme di visualizzazione grafica.
    Viene utilizzato per facilitare l'interpretazione e l'analisi delle informazioni, permettendo di identificare tendenze, \emph{pattern} e \emph{insight} 
    in modo chiaro e intuitivo, migliorandone la comprensione.}
}

\newglossaryentry{d3g} {
    name=D3.js,
    text=D3.js,
    sort=d3,
    description={Libreria JavaScript \emph{open source} usata per la creazione, a partire da dati organizzati, di visualizzazioni interattive che siano visibili attraverso un comune browser.
    Il suo nome, infatti, sta per \emph{Data-Driven Documents} (documenti basati sui dati). 
    In particolare, consente di creare grafici interattivi e dinamici sul web, utilizzando HTML, SVG e CSS}
}

\newglossaryentry{ictg} {
    name=\glslink{ict}{ICT},
    text=Information and Communication Technology,
    sort=ict,
    description={descrizione}
}

\newglossaryentry{erpg} {
    name=\glslink{erp}{ERP},
    text=Enterprise Resource Planning,
    sort=erp,
    description={descrizione}
}

\newglossaryentry{ebusinessg} {
    name=e-business,
    text=e-business,
    sort=ebusiness,
    description={descrizione}
}

\newglossaryentry{businessintg} {
    name=business intelligence,
    text=business intelligence,
    sort=businessintelligence,
    description={descrizione}
}

\newglossaryentry{serverfarmg} {
    name=server farm,
    text=server farm,
    sort=serverfarm,
    description={descrizione}
}

\newglossaryentry{mlg} {
    name=Machine Learning,
    text=Machine Learning,
    sort=machinelearning,
    description={descrizione}
}

\newglossaryentry{htmlg} {
    name=\glslink{html}{HTML},
    text=Hypertext Markup Language,
    sort=html,
    description={descrizione}
}

\newglossaryentry{cssg} {
    name=\glslink{css}{CSS},
    text=Cascading Style Sheets,
    sort=css,
    description={descrizione}
}

\newglossaryentry{jsg} {
    name=\glslink{js}{JS},
    text=JavaScript,
    sort=js,
    description={descrizione}
}

\newglossaryentry{llmg} {
    name=\glslink{llm}{LLM},
    text=Large Language Model,
    sort=llm,
    description={Tipologia di modello di AI progettato per comprendere e
    generare testo con una capacità simile a quella umana; utilizzando reti neurali
    profonde, questi modelli apprendono regole linguistiche, coerenza del contesto e
    stili di scrittura da grandi quantità di dati testuali}
}

\newglossaryentry{infoloadg} {
    name=Information Overload,
    text=Information Overload,
    sort=informationoverload,
    description={descrizione}
}

\newglossaryentry{bigdatag} {
    name=Big Data,
    text=Big Data,
    sort=bigdata,
    description={descrizione}
}

\newglossaryentry{datascienceg} {
    name=Data Science,
    text=Data Science,
    sort=datascience,
    description={descrizione}
}

\newglossaryentry{chartjunkg} {
    name=chartjunk,
    text=chartjunk,
    sort=chartjunk,
    description={descrizione}
}

\newglossaryentry{sistemadiregoleg} {
    name=sistema di regole,
    text=sistema di regole,
    sort=sistemadiregole,
    description={descrizione}
}

\newglossaryentry{promptengg} {
    name=Prompt Engineering,
    text=Prompt Engineering,
    sort=promptengineering,
    description={descrizione}
}

\newglossaryentry{fslg} {
    name=\glslink{fsl}{FSL},
    text=Few-Shot Learning,
    sort=fsl,
    description={descrizione}
}

\newglossaryentry{iclg} {
    name=\glslink{icl}{ICL},
    text=In-Context Learning,
    sort=icl,
    description={descrizione}
}

\newglossaryentry{llama7bg} {
    name=llama 7b,
    text=llama 7b,
    sort=llama7b,
    description={descrizione}
}

\newglossaryentry{llamacppg} {
    name=llama.cpp,
    text=llama.cpp,
    sort=llamacpp,
    description={descrizione}
}

\newglossaryentry{motoreregoleg} {
    name=motore di regole,
    text=motore di regole,
    sort=motorediregole,
    description={descrizione}
}

\newglossaryentry{jsonrulesg} {
    name=json-rules-engine,
    text=json-rules-engine,
    sort=jsonrulesengine,
    description={descrizione}
}

\newglossaryentry{jsong} {
    name=JSON,
    text=JSON,
    sort=json,
    description={descrizione}
}

\newglossaryentry{expdataanalysisg} {
    name=Exploratory Data Analysis,
    text=Exploratory Data Analysis,
    sort=exploratorydataanalysis,
    description={descrizione}
}

\newglossaryentry{grammaticag} {
    name=grammatica,
    text=grammatica,
    sort=grammatica,
    description={descrizione}
}

\newglossaryentry{vifg} {
    name=\glslink{vif}{VIF},
    text=Visual Information Flow,
    sort=vif,
    description={descrizione}
}


\newglossaryentry{mbtig} {
    name=\glslink{mbti}{MBTI},
    text=Myers-Briggs Type Indicator,
    sort=myersbriggstypeindicator,
    description={descrizione}
}

\newglossaryentry{wcagg} {
    name=\glslink{wcag}{WCAG},
    text=Web Content Accessibility Guidelines,
    sort=webcontentaccessibilityguidelines,
    description={descrizione}
}

\newglossaryentry{svgg} {
    name=\glslink{svg}{SVG},
    text=Scalable Vector Graphics,
    sort=scalablevectorgraphics,
    description={descrizione}
}


\makeglossaries

\bibliography{appendix/bibliography}

\defbibheading{bibliography} {
    \cleardoublepage
    \phantomsection
    \addcontentsline{toc}{chapter}{\bibname}
    \chapter*{\bibname\markboth{\bibname}{\bibname}}
}

% Spacing between entries
\setlength\bibitemsep{1.5\itemsep}

\DeclareBibliographyCategory{opere}
\DeclareBibliographyCategory{web}

\addtocategory{opere}{womak:lean-thinking}
\addtocategory{web}{site:agile-manifesto}

\defbibheading{opere}{\section*{Riferimenti bibliografici}}
\defbibheading{web}{\section*{Siti Web consultati}}


\captionsetup{
    tableposition=top,
    figureposition=bottom,
    font=small,
    format=hang,
    labelfont=bf
}

\hypersetup{
    %hyperfootnotes=false,
    %pdfpagelabels,
    colorlinks=true,
    linktocpage=true,
    pdfstartpage=1,
    pdfstartview=,
    breaklinks=true,
    pdfpagemode=UseNone,
    pageanchor=true,
    pdfpagemode=UseOutlines,
    plainpages=false,
    bookmarksnumbered,
    bookmarksopen=true,
    bookmarksopenlevel=1,
    hypertexnames=true,
    pdfhighlight=/O,
    %nesting=true,
    %frenchlinks,
    urlcolor=webbrown,
    linkcolor=RoyalBlue,
    citecolor=webgreen
    %pagecolor=RoyalBlue,
}

% Delete all links and show them in black
\if \isprintable 1
    \hypersetup{draft}
\fi

% Listings setup
\lstset{
    language=[LaTeX]Tex,%C++,
    keywordstyle=\color{RoyalBlue}, %\bfseries,
    basicstyle=\small\ttfamily,
    %identifierstyle=\color{NavyBlue},
    commentstyle=\color{Green}\ttfamily,
    stringstyle=\rmfamily,
    numbers=none, %left,%
    numberstyle=\scriptsize, %\tiny
    stepnumber=5,
    numbersep=8pt,
    showstringspaces=false,
    breaklines=true,
    frameround=ftff,
    frame=single
}

\definecolor{webgreen}{rgb}{0,.5,0}
\definecolor{webbrown}{rgb}{.6,0,0}

\newcommand{\sectionname}{sezione}
\addto\captionsitalian{\renewcommand{\figurename}{Figura}
                       \renewcommand{\tablename}{Tabella}}

\newcommand{\glsfirstoccur}{\ap{{[g]}}}

\newcommand{\intro}[1]{\emph{\textsf{#1}}}

% Risks environment
\newcounter{riskcounter}                % define a counter
\setcounter{riskcounter}{0}             % set the counter to some initial value

%%%% Parameters
% #1: Title
\newenvironment{risk}[1]{
    \refstepcounter{riskcounter}        % increment counter
    \par \noindent                      % start new paragraph
    \textbf{\arabic{riskcounter}. #1}   % display the title before the content of the environment is displayed
}{
    \par\medskip
}

\newcommand{\riskname}{Rischio}

\newcommand{\riskdescription}[1]{\textbf{\\Descrizione:} #1.}

\newcommand{\risksolution}[1]{\textbf{\\Soluzione:} #1.}

% Use case environment
\newcounter{usecasecounter}             % define a counter
\setcounter{usecasecounter}{0}          % set the counter to some initial value

%%%% Parameters
% #1: ID
% #2: Nome
\newenvironment{usecase}[2]{
    \renewcommand{\theusecasecounter}{\usecasename #1}  % this is where the display of
                                                        % the counter is overwritten/modified
    \refstepcounter{usecasecounter}             % increment counter
    \vspace{10pt}
    \par \noindent                              % start new paragraph
    {\large \textbf{\usecasename #1: #2}}       % display the title before the
                                                % content of the environment is displayed
    \medskip
}{
    \medskip
}

\newcommand{\usecasename}{UC}

\newcommand{\usecaseactors}[1]{\textbf{\\Attori Principali:} #1. \vspace{4pt}}
\newcommand{\usecasepre}[1]{\textbf{\\Precondizioni:} #1. \vspace{4pt}}
\newcommand{\usecasedesc}[1]{\textbf{\\Descrizione:} #1. \vspace{4pt}}
\newcommand{\usecasepost}[1]{\textbf{\\Postcondizioni:} #1. \vspace{4pt}}
\newcommand{\usecasealt}[1]{\textbf{\\Scenario Alternativo:} #1. \vspace{4pt}}

% Namespace description environment
\newenvironment{namespacedesc}{
    \vspace{10pt}
    \par \noindent  % start new paragraph
    \begin{description}
}{
    \end{description}
    \medskip
}

\newcommand{\classdesc}[2]{\item[\textbf{#1:}] #2}

% Code in Latex
% CSS
\lstdefinelanguage{CSS}{
    keywords={color,background-image:,margin,padding,font,weight,display,position,top,left,right,bottom,list,style,border,size,white,space,min,width, transition:, transform:, transition-property, transition-duration, transition-timing-function},	
    sensitive=true,
    morecomment=[l]{//},
    morecomment=[s]{/*}{*/},
    morestring=[b]',
    morestring=[b]",
    alsoletter={:},
    alsodigit={-}
}

% JavaScript
\lstdefinelanguage{JavaScript}{
    morekeywords={typeof, new, true, false, catch, function, return, null, catch, switch, var, if, in, while, do, else, case, break},
    morecomment=[s]{/*}{*/},
    morecomment=[l]//,
    morestring=[b]",
    %morestring=[b]',       %per apostrofi
    morestring=[b]`
}

\lstdefinelanguage{HTML5}{
    language=html,
    sensitive=true,	
    alsoletter={<>=-},	
    morecomment=[s]{<!-}{-->},
    tag=[s],
    otherkeywords={
    % General
    >,
    % Standard tags
    <!DOCTYPE,
    </html, <html, <head, <title, </title, <style, </style, <link, </head, <meta, />,
    % body
    </body, <body,
    % Divs
    </div, <div, </div>, 
    % Paragraphs
    </p, <p, </p>,
    % scripts
    </script, <script,
    % More tags...
    <canvas, /canvas>, <svg, <rect, <animateTransform, </rect>, </svg>, <video, <source, <iframe, </iframe>, </video>, <image, </image>, <header, </header, <article, </article
    },
    ndkeywords={
    % General
    =,
    % HTML attributes
    charset=, src=, id=, width=, height=, style=, type=, rel=, href=,
    % SVG attributes
    fill=, attributeName=, begin=, dur=, from=, to=, poster=, controls=, x=, y=, repeatCount=, xlink:href=,
    % properties
    margin:, padding:, background-image:, border:, top:, left:, position:, width:, height:, margin-top:, margin-bottom:, font-size:, line-height:,
    % CSS3 properties
    transform:, -moz-transform:, -webkit-transform:,
    animation:, -webkit-animation:,
    transition:,  transition-duration:, transition-property:, transition-timing-function:,
    }
}

\definecolor{editorGray}{rgb}{0.95, 0.95, 0.95}
\definecolor{editorOcher}{rgb}{1, 0.5, 0} % #FF7F00 -> rgb(239, 169, 0)
\definecolor{editorGreen}{rgb}{0, 0.5, 0} % #007C00 -> rgb(0, 124, 0)

\lstdefinestyle{htmlcssjs} {%
  % General design
    backgroundcolor=\color{editorGray},
    basicstyle={\footnotesize\ttfamily},   
    frame=none,
    % line-numbers
    xleftmargin={0.75cm},
    numbers=left,
    stepnumber=1,
    firstnumber=1,
    numberfirstline=true,	
    % Code design
    identifierstyle=\color{black},
    keywordstyle=\color{blue}\bfseries,
    ndkeywordstyle=\color{editorGreen}\bfseries,
    stringstyle=\color{editorOcher}\ttfamily,
    commentstyle=\color{brown}\ttfamily,
    % Code
    language=HTML5,
    alsolanguage=JavaScript,
    alsodigit={.:;},	
    tabsize=2,
    showtabs=false,
    showspaces=false,
    showstringspaces=false,
    extendedchars=true,
    breaklines=true,
    % Accents
    literate=%
    {é}{{\'e}}1 
    {è}{{\`e}}1 
    {à}{{\`a}}1
    {ò}{{\`e}}1 
    {ù}{{\`u}}1
    {ì}{{\`i}}1 
}

\newcommand{\icon}[2][1cm]{\begin{minipage}{#1}\includegraphics[width=#1]{#2}\end{minipage}}