\cleardoublepage
\phantomsection
\pdfbookmark{Sommario}{Sommario}
\begingroup
\let\clearpage\relax
\let\cleardoublepage\relax
\let\cleardoublepage\relax

\chapter*{Sommario}

Il presente documento esibisce un resoconto dell'attività di stage condotta dal laureando \myName \space presso 
l'azienda Zucchetti S.p.A. (sede di Padova) per una durata di circa trecento ore.\\

\noindent Esso si propone di esaminare il campo della \emph{data visualization} e delle infografiche, riportandone 
un'applicazione pratica effettuata tramite la libreria D3.js. 
L'obiettivo dello stage era, infatti, sviluppare un sistema di visualizzazione innovativo e interattivo 
per i risultati ottenuti da algoritmi basati su Intelligenza Artificiale.\\

\noindent La parte introduttiva del documento delinea le aspettative iniziali e i piani per lo stage, 
inquadrando il tutto nel contesto aziendale. \\
Successivamente, vengono esplorati i fondamenti della \emph{data visualization}, focalizzandosi in particolar modo 
sulle infografiche e sul contesto web.\\
In seguito, si approfondisce, attraverso un caso pratico, la creazione di infografiche interattive 
realizzate mediante la libreria D3.js.\\
Infine, il documento si conclude con una riflessione sull'esperienza effettuata, inclusi i risultati ottenuti 
e le considerazioni personali sul percorso compiuto.\\

\noindent Per quanto riguarda il documento in sé, si riportano di seguito le convenzioni tipografiche adottate:
\begin{itemize}
    \item gli acronimi, le abbreviazioni e i termini ambigui o di uso non comune menzionati vengono definiti nel glossario, situato alla fine del presente documento;
    \item per la prima occorrenza dei termini riportati nel glossario viene utilizzata la seguente nomenclatura: \emph{parola}\glsfirstoccur;
    \item i termini in lingua straniera o facenti parti del gergo tecnico sono evidenziati con il carattere \emph{corsivo}.
\end{itemize}


%\vfill

%\selectlanguage{english}
%\pdfbookmark{Abstract}{Abstract}
%\chapter*{Abstract}

%\selectlanguage{italian}

\endgroup

\vfill
