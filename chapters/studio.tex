\chapter{Fondamenti e metodologie di data visualization}
\label{cap:studio}
\intro{In questo capitolo verranno descritti i principi chiave che regolano la visualazzazione dei dati. 
A tal riguardo, una particolare attenzione sarà rivolta alle infografiche e alla visualizzazione dei dati sul web.}\\

\section{Cos'è la data visualization e perchè è importante}
Definition of Data Visualization: What is data visualization?
History and Evolution: Brief history and evolution of data visualization.
Importance: Why is data visualization important in the current data-driven world?



With the development of statistics and
computer technology, and the increasing dependence
of businesses on big data-based decision-making,
data visualization has received more and more
attention.
There is a huge knowledge gap between non-expert
users and visual models when using technology to
translate information visually. Without classification,
it is difficult for users to choose effective techniques
to represent data.

L'ultima incarnazione dell'età dell'informazione è rappresentata dal fenomeno dei big data,
che stanno rivoluzionando sia i modi di ricognizione della realtà da parte edi governi, delle imprese e 
degli organi di informazione, sia le teniche di rappresentazione delle informazioni per trarre 
intelligenza dalla grande mole dei dati disponibili. Le strategie di raffigurazione dei dati permettono infatti
di cogliere relazioni ed evidenze che non si sarebbero manifestate senza la creazione di segni adeguati alla loro esposizione.
Esiste una tradizione secolare del pensiero filosofico che insiste sulla correlazione necessaria di pensiero 
e raffigurazione. Nell'età contemporanea, tuttavia, questa esigenza di rappresentazione dei dati
ha subito un incremento senza paragoni: il lavoro di raffigurazione non è richiesto soltanto per le intuizioni geniali, ma anche
per la gestione della vita ordinaria di qualsiasi impresa o pubblica amministrazione che voglia assumere decisioni ponderate
sui consumi di risorse e sugli effetti delle proprie strategie di sviluppo. Ma non solo, anche per autopresentarsi 
al pubblico, cercando di comunicaer la propria identità.
Per giornali e istituizioni formative, questo requisito è ancora più stringente, dal momento che il loro obiettivo è portare alla 
luce la struttura del reale per favorire la formazione della coscienza critica nei propri lettori e fruitori.

Epoca dei big dati => troppi ed eterogenei (informazione dell’eccesso), tra ampiezza del sapere sociale, motori di ricerca, volontà dlle persone di esprimersi e condividere (e.g. nei social media). 
Complicato dominarli in modo da spiegarne relazioni tra elementi catturati e eventuali conseguenze, risoluzioni e valutazioni.  
Intelligenza = attività di scoperta delle funzioni strutturali, nessi imprevisti tra i vari fenomeni e, al contempo, facoltà elabora gli atteggiamenti da tenere nei confronti di eventuali regolarità. 
Per afferrare ciò necessario poter fissare l’attenzione in segni maneggevoli, a colpo d’occhio, leggibile in modo semplice e suggestivo. Questo è obiettivo infografica.  In altre parole l’obiettivo base è trasformare dei dati in un’idea e raccontarla in maniera semplice e comprensibile

Infografica è dunque prodotto di intelligenza che interpreta dati in modo da inquadrare un fenomeno, situazione, evento o processo per assumere dele decisioni più ponderate.
Infografica è un prodotto di comunicazione, no imparziale e completa (in quel caso guardi i dati e basta), forma di racconto, di lettura che per forza di cose esclude delle connessioni presenti nel dataset, devo rintracciare le parti/leggi che risultano più interessanti per destinatario.



Data visualizations are surprisingly common in your everyday life, but they often appear in the form of well-known charts and graphs. A combination of multiple visualizations and bits of information is often referred to as infographics.

Data visualizations can be used to discover unknown facts and trends. You may see visualizations in the form of line charts to display change over time. Bar and column charts are useful when observing relationships and making comparisons. Pie charts are a great way to show parts of a whole. And maps are the best way to visually share geographical data.
What makes a good data visualization?


Good data visualizations are created when communication, data science, and design collide. Data visualizations done right offer key insights into complicated datasets in ways that are meaningful and intuitive. American statistician and Yale professor Edward Tufte believes excellent data visualizations consist of ‘complex ideas communicated with clarity, precision, and efficiency.’
(vedi immagine quella con venn)

In order to craft a good data visualization, you need to start with clean data that is well-sourced and complete. Once your data is ready to visualize, you need to pick the right chart. This can be tricky, but there are many resources available to help you choose the right type of chart for your data. 

After you have decided which chart type is best, you need to design and customize your visualization to your liking. Remember, simplicity is key – you don’t want to add any elements that distract from the data. Now that your visualization is complete, it’s time to publish and share it with your colleagues, customers, or readers.
Why does data visualization matter?

Better decision making
oday more than ever, organizations are using data visualizations, and data tools, to ask better questions and make better decisions. Emerging computer technologies and new user-friendly software programs have made it easy to learn more about your company and make better data-driven business decisions.

The strong emphasis on performance metrics, data dashboards, and Key Performance Indicators (KPIs) shows the importance of measuring and monitoring company data. Common quantitative information measured by businesses includes units or products sold, revenue by quarter, department expenses, employee stats, and company market share.

Meaningful storytelling

Data visualizations and information graphics (infographics) have become essential tools for today’s mainstream media.

Data journalism is on the rise and journalists consistently rely on quality visualization tools to help them tell stories about the world around us. Many well-respected institutions have fully embraced data-driven news including The New York Times, The Guardian, The Washington Post, Scientific American, CNN, Bloomberg, The Huffington Post, and The Economist.

Marketers also benefit greatly from the combination of quality data and emotional storytelling. Good marketers make data-driven decisions on a daily basis, but sharing with their customers requires a different approach – one that touches both intelligently and emotionally. Data visualizations help marketers share their message using statistics and heart.

Data literacy


Being able to understand and read data visualizations has become a necessary requirement for the 21st century. Because data visualization tools and resources have become readily available, more and more non-technical professionals are expected to be able to gather insights from data.

Increasing data literacy around the world has been one of the main pillars of Infogram’s mission from day one. We truly believe in the importance of data education and support around the world.
Infogram CEO Mikko Jarvenpaa explains:

“We believe that better-informed people make better decisions, and people who can both read and create data-driven communications are central to this.”

Why use data visualization?

    Make data easier to understand and remember
    Discover unknown facts, outliers, and trends
    Visualize relationships and patterns quickly
    Ask better questions and make better decisions

%Se aggiunta sta roba su biblio https://infogram.com/blog/what-is-data-visualization/?_gl=1*1rypyyw*_up*MQ..*_ga*MTg5MzY2NzU2MS4xNzE5MzMxMTgw*_ga_LD50PRQER7*MTcxOTMzMTE3OS4xLjAuMTcxOTMzMTE3OS4wLjAuMA..

\section{Come visualizzare i dati attraverso le infografiche}
Infographics Explained: What are infographics and how do they differ from other forms of data visualization?
Design Principles: Key principles for designing effective infographics (e.g., clarity, accuracy, aesthetics).
Process: Step-by-step process to create an infographic. (INFOMODEL)

An important difference is that a data visualization is just one (i.e. a map, graph, chart or diagram), while an infographic often contains multiple data visualizations. A second key difference is that infographics contain additional elements like narrative and graphics
Praticamente data viz è infografica SENZA STORYTELLING
% anche parte di classificazione

Fonte infografica è set di dati, solitamente disomogenei (x formato, tipo di contenuto, origine), bisogna formalizzare la comunicazione integrando elementi grafici con testuali per dare una lettura più semplice possibile del significato implicito del dataset.
The visual display of quantitative information, Edward Tufte [2001], Introduction, p.10 “un’infografica mostra visivamente grandezze misurate mediante l’uso combinato di punti, linee, un sistema di coordinate, numeri, simboli, parole, ombreggiature e colore” definizione da estendere con quanto detto sopra di comunicazione a destinatario (no solo focus su oggetto). 


\section{Visualizzare i dati sul web}
Web Technologies for Data Visualization: Overview of technologies used for web-based data visualization.
Cosa cambia da quello scritto sopra, importanza dell'interattività etc.
Popular Libraries and Tools: Discussion on popular libraries and tools (e.g., D3.js, Chart.js, Tableau).


\noindent Esempio di utilizzo di un termine nel glossario \\
\gls{api}. \\

\noindent Esempio di citazione in linea \\
\cite{site:agile-manifesto}. \\

\noindent Esempio di citazione nel pie' di pagina \\
citazione\footcite{womak:lean-thinking} \\