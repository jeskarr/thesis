\chapter{Data visualization}
\label{cap:studio_data_viz}
\intro{In questo capitolo verrà presentata una panoramica della data visualization, spiegandone caratteristiche, usi e principi.
Viene inoltre proposto un metodo di classificazione dei grafici per ottimizzare le visualizzazioni.}\\

\section{Definizione e applicazioni}

\subsection{La necessità di rappresentare i dati}
%INTRO + CAP 1: “Design della mente – infografica e data viz” di Paolo Bottazini, Michele Gotuzzo
La visualizzazione dei dati ha una tradizione millenaria. Sin dagli albori della civiltà umana, 
le persone hanno sviluppato metodi per rappresentare visivamente i dati al fine di renderli più facili da comprendere, 
memorizzare e trasmettere.
Negli ultimi decenni, tuttavia, questa esigenza ha conosciuto un incremento senza precedenti.

Viviamo infatti nell'epoca dell'``information overload'' e dei ``big data'', dove le informazioni
arrivano con una velocità, un volume e una varietà così travolgenti che non siamo in grado di comprenderli se non attraverso
un ulteriore strato di astrazione.
Tale rivoluzione si deve soprattutto allo sviluppo delle tecnologie digitali, grazie alle quali siamo arrivati a disporre di una quantità 
apparentemente infinita di informazioni, tra sapere sociale, motori di ricerca e volontà delle persone di esprimersi.

In questo contesto, la \emph{data visualization} è diventata fondamentale per interpretare e trarre intelligenza da questa grande mole di dati.
Nello specifico, essa consente di fissare l'attenzione in segni maneggevoli - le rappresentazioni grafiche - che consentono una lettura dell'informazione
semplice e intuitiva. Tramite tali segni, è possibile cogliere funzioni strutturali e relazioni impreviste tra fenomeni, oltre che scoprire regolarità che 
altrimenti sarebbero rimaste nascoste, facilitando eventuali risoluzioni e valutazioni.

\bigskip
\noindent Possiamo dunque definire la \emph{data visualization} come l'insieme dei meccanismi di traduzione dei dati in rappresentazioni grafiche
mediante i quali l'informazione viene resa più chiara ed efficace. Più specificamente, 
potremmo dire che è un processo di traslazione dei dati all'interno di un contesto visivo atto ad amplificarne la cognizione.

\subsection{Obiettivi della data visualization}
%cap 6: https://onlinelibrary.wiley.com/doi/book/10.1002/9781119209560
Gli obiettivi alla base della \emph{data visualization} sono:
\begin{itemize}
    \item \textbf{Identificare}, ovvero trovare un dataset significativo, né troppo banale né troppo complesso, da cui sia possibile
    ricavarne delle informazioni utili;
    \item \textbf{Manipolare}, ossia analizzare e combinare i dati in modo da chiarirne il significato;
    \item \textbf{Formattare}, ovvero standardizzare il modo con cui si accede ai dati per una consumazione più efficiente;
    \item \textbf{Presentare}, ossia rappresentare i dati formattati in modo da esprimerne il significato nascosto.
\end{itemize}
\noindent Si desidera dunque massimizzare l'efficacia ed efficienza della visualizzazione dell'informazione. Ciò comporta:
\begin{itemize}
    \item Per quanto riguarda l'efficienza: ridurre la complessità e il rumore e nella visualizzazione che sono non necessari e potrebbero
    portare addirittura a correlazioni incorrette.
    \item Per quanto riguarda l'efficacia: fornire informazioni comprensibili e utili in modo che sia possibile prendere decisioni informate in base a esse.
\end{itemize}

\bigskip
%https://infogram.com/blog/what-is-data-visualization/?_gl=1*1rypyyw*_up*MQ..*_ga*MTg5MzY2NzU2MS4xNzE5MzMxMTgw*_ga_LD50PRQER7*MTcxOTMzMTE3OS4xLjAuMTcxOTMzMTE3OS4wLjAuMA..
\noindent Il fine ultimo di questi obiettivi è quello di:
\begin{itemize}
    \item Rendere i dati comprensibili e memorabili,
    \item Facilitare nuove scoperte e individuare trend e valori anonimi,
    \item Visualizzare velocemente relazioni e regolarità nei dati,
    \item Rendere migliore e più consapevole la presa di decisioni e stimolare la formulazione di ulteriori domande più specifiche.
\end{itemize} 


\subsection{Applicazioni}
Nell'età contemporanea, il processo di visualizzazione dei dati è ormai diventato essenziale per la gestione quotidiana di qualsiasi 
impresa, governo e organo informativo.

Imprese e pubbliche amministrazioni sfruttano la \emph{data visualization} per comprendere meglio la propria organizzazione e 
prendere decisioni più informate basate sui dati, ad esempio per quanto riguarda la gestione delle risorse e lo sviluppo di strategie future. 
Questo approccio, inoltre, è cruciale anche per la percezione esterna dell'organizzazione e dunque per contraddistinguersi da enti a lei analoghi.

Alla stessa maniera, anche per gli organi informativi (e.g. i giornali) la \emph{data visualization} è diventata indispensabile. 
Infatti, solo grazie a essa è possibile ricavare delle intuizioni altrimenti impossibili e rendere fruibili 
tali risultati in maniera chiara e intuitiva. Ciò risulta particolarmente importante per tali organi in quanto la loro mission è proprio diffondere 
e comunicare informazioni.

\section{Principi e linee guida}
%“The visual Display of Quantitative Information” di Edward R. Tufte
%CAP 1: Graphical excellence
Principi generali della graphical excellence
-	Idee complesse comunicate con chiarezza, precisione ed efficienza (una rappresentazione elegante combina semplicità di design e complessità dei dati)
-	L'utente finale deve avere il maggior numero di idee nel periodo più corto possibile con il minore uso di inchiostro (togliere dunque tutte le parti che possono essere eliminate senza perdere dati) nello spazio più piccolo
-	Quasi sempre multivariata 
-	Bisogna dire la verità sui dati
%CAP 1: Graphical Integrity
Principi generali dell'integrità grafica
-	La rappresentazione di numeri misurate fisicamente sul grafico dovrebbe essere direttamente proporzionale alle quantità numeriche rappresentate
-	Usare etichette dettagliate e chiare per evitare distorsione/ambiguità dei grafici (per documentare eventi importanti nei dati, spiegare i dati nel grafico stesso)
-	Mostrare la vera variazione dati non la variazione nel design (e.g. usare scale con intervalli regolari)
-	Nelle visualizzazioni di soldi nel tempo, unità standardizzati sono meglio che unità nominali (e.g. per non essere influenzati dall'inflazione etc.)
-	Il numero di variabili/dimensioni non deve eccede il numero di queste nei dati (e.g. non mostrare una variazione di una dimensione con un'area, tipo banconota che si ingrandisce diminuisce per rappresentare l'inflazione non la rappresenta giustamente perché l'area è a due dimensioni mentre il problema riguarda 1)
-	Grafici non dovrebbero citare dati fuori dal loro contesto
%CAP 8: Data density and small multiples
Densità = number entries in una matrice di dati / area del grafico.
Meglio se è più alta (mappe è altissima), risulta anche essere più affidabile. Bisogna comunque evitare di sovrappopolare lo spazio, in quel caso conviene utilizzare tecniche di data-reduction (averaging, clustering, smoothing).
Attenzione! Stiamo parlando della parte di dati, per quanto riguarda la parte non strettamente riguardante questi (chartjunk) meglio meno.
%CAP 9: AESTHETICS AND TECHNIQUE IN DATA GRAPHICAL DESIGN
Per migliorare il design di info statistiche bisogna:
-	Scegliere un format e un design appropriato
    o	Combinazione frasi, tabelle e grafici (non per forza tutto ma di solito almeno un paio):
        -	Frasi non vanno bene per più di due numeri perché non ottimali per comunicare comparazione, in quel caso conviene distribuirli su una tabella (sempre se rimangono data set piccoli, meglio della torta da non usare praticamente!)
        -	Tabelle funzionano bene anche quando ho tanti confronti tra elementi localizzati (supertable, divisa tipo in paragrafi per topic, meglio di tanti piccoli bar chart)
        -	Per numeri con tanti dati testuali, grafico con molte etichette.
-	Usare parole, numeri e disegni assieme con coerenza
-	Riflettere un bilancio, proporzione, un senso di una certa rilevanza
-	Visualizzare una complessità (anche dei dettagli) che sia accessibile
OK	NON OK
Evitati misteriosità, troppa elaborazione. Le parole sono esplicitate	Troppe abbreviazioni, l'utente deve passare tutto il testo per capire
Parole da sx a dx (per occidente)	Parole in verticale (Y-axis) o in altre direzioni
Piccoli messaggi che aiutano a spiegare i dati	La grafica è criptica, ha bisogno di più sguardi a testo sparso
Evitati sfumature, colori e tratteggi incrociati troppo elaborati; meglio nessuna legenda necessaria	Codifica strana che richiede continuare a controllare grafico e legenda
Grafico che attrae l'utente, che lo incuriosisce	Il grafico è pieno di chartjunk
Colori, se usati, devono essere scelti in modo che le persone color-deficient e color-blind possano decifrarlo	Usare verde e rosso come contrasto essenziale e design che non considera utenti color-deficient 
Tipo di carattere chiaro, preciso, modesto	Caratteri troppo tracotanti
Maiuscolo e minuscolo con serif	Caratteri tutti maiuscoli, sans serif
-	Avere una storia da raccontare sui dati
-	Essere disegnato in maniera professionale, con dettagli tecnici
-	Evitare decorazioni content-free (chartjunk)


%“L'arte funzionale – Infografica e visualizzazione delle informazioni” di Alberto Cairo
% cap 2
“La funzione vincola la forma” dell'infografica, i.e. deve aiutarci a compiere alcune attività intellettive (al contrario dell'arte in sé) e pertanto deve aspirare a oggettività, precisione e funzionalità. 
“La forma segue la funzione” Louis Sullivan sull'architettura.
Un complesso di dati può assumere differenti forme ma non può assumere una forma qualsiasi, in particolare meglio sono definiti gli scopi di un artefatto, meno vasta sarà la gamma di forme che può assumere.
Umano non bravo a calcolare le aree delle superfici, bensì è più bravo a confronta singole dimensioni (come lunghezza o altezza).
%cap 3
Tufte per efficienza grafica:
-	Rapporto dati-inchiostro: inchiostro x codifica dati / tot. dell'inchiostro utilizzato per stampare il grafico (più vicino a 1 meglio è)
Tuttavia, elemento che Tufte considera “chartjunk” non è sempre da buttare, infatti in alcuni casi sembra facilitare la memoria. Idea di Holmes che finchè la funzione principale è comunicare i dati per il resto ci si può divertire con la forma con cui compaiono le statistiche
Neurath -> “umanizzazione della conoscenza” , linugaggio universale basato su pittogrammi per superare barriere culturali. In “From Hyeroglyphics to Isotype” dice “il soggetto della mostra deve essere serio ma va combinato con un fascino e un richiamo diretto per tutti”.
%Cap 4
Per capire quale meglio tra Tufte e Homes+Neurath, bisogna ricordare che infografica ha come obiettivo presentare le info e permettere all'utente di esplorarle.
In ogni caso, che sia verso la parte più minimalistica di Tufte che quella più artistica di Holmes, bisogna sfruttare sempre lo spazio a disposizione per ricercare la profondità (nei limiti del ragionevole) e solo poi ci si può preoccupare di come rendere più bella la presentazione (gli effetti speciali occupano uno spazio che avrebbe potuto essere utilizzato per evidenziare altri aspetti del fenomeno e troppo spazio non deve essere sprecato per elementi che non aiutano i lettori a capire i dati). 
Profondità da ricercare sempre e non avere sfiducia nell'interesse dei lettori per gli argomenti trattati.
Le infografiche non dovrebbero semplificare le informazioni/messaggio bensì chiarirlo evidenziando tendenze, schemi o realtà prima non visibili (devono permettere riflessioni e non essere superficiali, sostituibili con una riga di testo).


%https://infogram.com/blog/what-is-data-visualization/?_gl=1*1rypyyw*_up*MQ..*_ga*MTg5MzY2NzU2MS4xNzE5MzMxMTgw*_ga_LD50PRQER7*MTcxOTMzMTE3OS4xLjAuMTcxOTMzMTE3OS4wLjAuMA..
In order to craft a good data visualization, you need to start with clean data that is well-sourced and complete. Once your data is ready to visualize, you need to pick the right chart. This can be tricky, but there are many resources available to help you choose the right type of chart for your data. 
After you have decided which chart type is best, you need to design and customize your visualization to your liking. Remember, simplicity is key – you don't want to add any elements that distract from the data. Now that your visualization is complete, it's time to publish and share it with your colleagues, customers, or readers.

Good data visualizations are created when communication, data science, and design collide. Data visualizations done right offer key insights into complicated datasets in ways that are meaningful and intuitive. American statistician and Yale professor Edward Tufte believes excellent data visualizations consist of ‘complex ideas communicated with clarity, precision, and efficiency.'
(vedi immagine quella con venn)

%cap1: “L'arte funzionale – Infografica e visualizzazione delle informazioni” di Alberto Cairo
Gerarchia DIKV (Data, Information, Knowledge, Wisdom) -> vedi foto (realtà -> documentazione di osservazioni -> codifica -> consumo di informazione -> comprensione profonda della conoscenza acquisita). Il cervello cerca sempre di ridurre la distanza tra fenomeni e conoscenza (cognizione), il ruolo di un architetto dell'info è anticipare questo processo e generare ordine prima che il cervello delle persone lo faccia da solo.


\section{Classificare i grafici}
Because data visualization tools and resources have become readily available, more and more non-technical professionals are expected to be able to gather insights from data.

There is a huge knowledge gap between non-expert
users and visual models when using technology to
translate information visually. Without classification,
it is difficult for users to choose effective techniques
to represent data.

%file criteri_scelta_grafico_v0.1.0 + codice + prompt LLM
%fonte: https://www.data-to-viz.com/ e https://ft-interactive.github.io/visual-vocabulary/




