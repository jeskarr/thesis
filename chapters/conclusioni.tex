\chapter{Conclusioni}
\label{cap:conclusioni}
\intro{In questo capitolo verranno delineati i risultati raggiunti durante lo stage e 
verranno fornite alcune considerazioni finali.}\\

\section{Rendiconto dei risultati}
\subsection{Vincoli e obiettivi generali}
I vincoli temporali e tecnologici del progetto di stage sono stati rispettati. 
Infatti, sono state effettuate un totale di 306 ore di lavoro e si sono utilizzati, tra le altre tecnologie, 
\gls{html}, \gls{css}, \gls{js} e \gls{d3g} per lo sviluppo pratico di un'infografica.

\bigskip
\noindent Per quanto riguarda gli obiettivi generali, essi sono stati raggiunti pienamente.

Infatti, lo stagista ha acquisito una solida comprensione del campo 
della \gls{datavizg} e delle infografiche, riflessa nei capitoli centrali del presente documento. 

Inoltre, è stata acquisita una competenza avanzata nell'uso di \gls{d3g}, attraverso la lettura di risorse di studio e 
attraverso l'applicazione pratica, concretizzata tramite lo sviluppo di elementi di \gls{datavizg} sia presenti nell'infografica sia  
indipendenti da essa, realizzati a scopo auto-formativo.

Infine, sono state rafforzate anche le competenze di sviluppo web dello stagista, tramite la creazione in \gls{html}, \gls{css} e \gls{js} dell'esempio di infografica 
descritto nel capitolo precedente.

\subsection{Attività svolte e obiettivi specifici}
Lo sviluppo del progetto di stage ha portato ad approfondire e analizzare alcuni aspetti inizialmente non previsti o sottovalutati. 
Ciò ha comportato il prolungamento di alcune attività e, dunque, la riduzione del tempo disponibile per le successive.

Nello specifico, per quanto riguarda l'attività di ``Identificazione dei criteri per la scelta del tipo di grafico più adatto, considerando la struttura dei
dati e gli obiettivi della visualizzazione'', oltre all'identificazione in sé si è sviluppato lo strumento \emph{Chart-chooser} per automatizzare tale processo. 
L'attività ``Ricerca approfondita di informazioni e guide riguardanti il concetto di infografica'', invece, ha rivelato un aspetto nella creazione di infografiche che 
non era stato considerato inizialmente, ovvero l'importanza della storia di un'infografica. Si è dunque sentita l'esigenza di sviluppare uno strumento, 
\emph{Infographic-helper}, che aiutasse gli utenti nella costruzione di tale storia.

Lo sviluppo di tali strumenti ha comportato uno slittamento delle attività successive, impedendo la completa realizzazione dell'attività di 
``Implementazione dei metodi di validazione individuati''. 

\bigskip
\noindent Per quanto riguarda gli obiettivi specifici, si riporta di seguito il loro stato di soddisfacimento alla fine del progetto:
\begin{table}[H]
    \centering
    \begin{tabular}{|>{\centering\arraybackslash} m{0.17\columnwidth} |>{\centering\arraybackslash} m{0.14\columnwidth} |>{\centering\arraybackslash} m{0.39\columnwidth}| >{\centering\arraybackslash} m{0.17\columnwidth}|}
        \hline
        \rowcolor{gray!20}
        \textbf{Identificativo} & \textbf{Tipo} & \textbf{Descrizione dell'obiettivo} & \textbf{Stato}\\
        \hline
        \textbf{O01} & Obbligatorio & Comprendere i principi fondamentali della \gls{datavizg}. & Soddisfatto \\
        \hline
        \textbf{O02} & Obbligatorio & Selezionare il grafico più adatto per tipo di struttura dati e obiettivo della visualizzazione. & Soddisfatto \\
        \hline
        \textbf{O03} & Obbligatorio & Realizzare grafici informativi e interattivi utilizzando \gls{d3g}. & Soddisfatto \\
        \hline
        \textbf{O04} & Obbligatorio & Realizzare un'infografica web funzionante in \gls{html}, \gls{css} e \gls{js} che visualizzi i risultati degli algoritmi di intelligenza artificiale e \gls{mlg}. & Soddisfatto \\
        \hline
        \textbf{O05} & Obbligatorio & Creare un \emph{template} dell'infografica. & Soddisfatto\tablefootnote{Vengono identificati dei ``template'' generali per i diversi obiettivi delle infografiche,  
        risultanti più utili di un singolo template relativo all'esempio pratico di infografica sviluppato.}\\
        \hline
        \textbf{O06} & Obbligatorio & Individuare dei metodi di validazione dell'infografica. & Soddisfatto \\
        \hline
        \textbf{D01} & Desiderabile & Sviluppare infografiche alternative. & Non soddisfatto \\
        \hline
        \textbf{D02} & Desiderabile & Implementare un \emph{chatbot} con \gls{llm} che interagisca con l'utente sull'infografica. & Soddisfatto parzialmente, non si utilizza \gls{llm}\\
        \hline
    \end{tabular}
    \vspace{0.2cm}
    \caption{Stato di soddisfacimento degli obiettivi a fine stage}
    \label{tab:obiettivi_stato}
\end{table}

\section{Considerazioni finali}
Pur non avendo potuto completare tutte le attività previste, lo stagista si ritiene soddisfatto dello stage svolto sia a livello lavorativo che umano. 

Tutti i colleghi, infatti, si sono dimostrati disponibili, rendendo possibile lavorare in un ambiente sereno e amichevole. 

Dal punto di vista tecnico, lo stage ha concesso l'opportunità di arricchire il proprio bagaglio di competenze, sia 
consolidando quanto imparato durante il corso di laurea sia apprendendo nuove conoscenze.
Le basi fornite dal corso, tuttavia, hanno facilitato l'acquisizione di queste nuove competenze, rendendo il processo di apprendimento più agevole.

Inoltre, lo stage ha contribuito a colmare anche alcune lacune pratiche, come la gestione degli orari lavorativi e l'organizzazione della ricerca e auto-formazione. 