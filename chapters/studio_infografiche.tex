\chapter{Infografiche}
\label{cap:studio_infografiche}
\intro{In questo capitolo verranno esaminate le caratteristiche funzionali e di design delle infografiche e 
presentate le loro applicazioni. Verrà inoltre fornita una classificazione delle infografiche basata sulla 
loro struttura e output visivo.}\\

\section{Definizione e applicazioni}
\subsection{Cos'è un infografica e in cosa si distingue dalla data visualization}
%cap1: Design della mente – infografica e data viz” di Paolo Bottazini, Michele Gotuzzo 
Analogamente alla \gls{datavizg}, le infografiche si occupano di rappresentare visivamente i dati al fine di comunicarli in maniera più semplice e accessibile.
Esse trasformano i dati in idee chiare, permettendo di inquadrare facilmente un fenomeno, una situazione o un processo, coadiuvando così la presa di decisioni ponderate sul tema presentato.
Queste idee sono generate a partire dall'interpretazione, formalizzazione e contestualizzazione dei dati, che spesso sono disomogenei per formato, tipo di contenuto e origine. Tale interpretazione ha il compito di 
svelare il significato nascosto nel \emph{dataset} e selezionare le connessioni o fenomeni più interessanti per il destinatario della comunicazione.
In altre parole, l'infografica non si limita a mostrare i dati, ma li organizza e li presenta in un modo da renderne più evidente significato e importanza.

Per raggiungere questo obiettivo, le infografiche combinano elementi testuali e grafici per presentare le informazioni; usando le parole di E. Tufte ``un'infografica mostra visivamente grandezze misurate mediante 
l'uso combinato di punti, linee, un sistema di coordinate, numeri, simboli, parole, ombreggiature e colore''. Pertanto, le infografiche rigettano la tradizione di ``purezza grafica'' (i.e. rappresentazione dei dati 
basata solo su grafici, diagrammi e mappe) imposta alla \gls{datavizg} e abbracciano, invece, un insieme molto più amplio di tecniche di comunicazione visiva, come illustrazioni, immagini, etc.
% TODO: aggiungere citazione come footnote The visual display of quantitative information, Edward Tufte [2001], Introduction, p.10
Inoltre, il focus dell'infografica non è solamente l'oggetto da rappresentare, come invece accade con la \gls{datavizg}. Infatti, le infografiche
prendono in considerazione anche il destinatario della comunicazione, scegliendo e disponendo gli elementi per costruire una narrazione.
Possiamo infatti pensare all'infografica come a una forma di racconto, di lettura che non è di per sé né imparziale né completa, ma si limita a fare luce ed esporre i punti salienti individuati nel \emph{dataset}.
%intro: “L'arte funzionale – Infografica e visualizzazione delle informazioni” di Alberto Cairo
In realtà, tale caratteristica delle infografiche è proprio ciò che maggiormente le distingue dalla \gls{datavizg}. Infatti, mentre le prime sono progettate per presentare i dati raccontandoli attraverso storie specifiche, guidando il lettore con una narrazione, la seconda
è più orientata verso l'esplorazione autonoma delle informazioni da parte del fruitore, che deve analizzare e scoprire i dati per conto proprio. 
Questa differenza nel livello di analisi richiesto si ripercuote anche sul tipo di pubblico di ciascun tipo di rappresentazione. Nello specifico, le infografiche, grazie alla loro capacità di semplificare e contestualizzare i dati, sono accessibili anche a coloro
che non possiedono capacità analitiche avanzate e, pertanto, hanno un pubblico più ampio. 

%intro: “L'arte funzionale – Infografica e visualizzazione delle informazioni” di Alberto Cairo
Nonostante queste differenze, infografiche e \gls{datavizg} sono strettamente connesse e complementari. Le infografiche, pur essendo maggiormente orientate alla narrazione, 
possono includere elementi che stimolano l'esplorazione autonoma dei dati. Allo stesso modo, le \emph{visualizzazioni dei dati} possono contenere aspetti narrativi che facilitano 
l'interpretazione.


\subsection{Applicazioni e vantaggi}
Le infografiche sono strumenti versatili che vengono utilizzati in una vasta gamma di settori e contesti, in particolare al fine 
di analizzare o presentare dati. Offrono infatti vantaggi significativi che le rendono particolarmente adatte a questi scopi.

%Infographics: The New Communication Tools in Digital Age, Waralak V. Siricharoen
Per quanto riguarda l'\textbf{analisi dei dati}, le infografiche sono utili per la cosiddetta \gls{expdataanalysisg}, dove il fruitore non ha una domanda precisa a cui sta cercando risposta, bensì 
vuole semplicemente scoprire cosa emerge di interessante dal \emph{dataset}.

Per quanto riguarda invece la \textbf{presentazione dei dati}, va da sé che le infografiche siano perfette per tale scopo, in quanto riescono attraverso una narrazione a comunicare in maniera efficace 
e abbracciare in tal modo un pubblico molto amplio.

\bigskip
\noindent In generale, quale sia l'applicazione, l'uso di infografiche consente anche di:
\begin{itemize}
    \item \textbf{Facilitare la scoperta di informazioni} nascoste e \textbf{stimolare la curiosità} per una ricerca futura;
    \item \textbf{Semplificare la comprensione}, facilitando così il raggiungimento della \emph{saggezza};
    \item \textbf{Raccontare una storia} che permetta una visione d'insieme della situazione, fenomeno o processo presentato.
\end{itemize}


\section{Classificare le infografiche}
% Acquired Codes of Meaning in Data Visualization and Infographics: Beyond Perceptual Primitives, Lydia Byrne, Daniel Angus, and Janet Wiles
Data visualization is grounded in the tradition of graphic purity, while infographics have no such ideal, and in contrast have traditionally made use of illustration.
 suggest that effective designs will make use of conventions and figurative elements to reduce the effort required for the user to understand and remember the message a visualization.
Data
visualizations often combined many facets of a subject into a single
graphic composition. Infographics, which are grounded in a tradition
of narrative, are far more likely in our sample to use a panel
composition where different component representations can be read
in sequence. Each component representation in a panel composition
may itself be a simple statistical graph

% file criteri_scelta_infografica_v0.1.0

\section{Design delle infografiche}
% fattori che influenzano la presentazione dei dati:
% per caratteri e colori guarda qaunto ho scritto su data viz.docs
% per titolo, boh (https://www.copypress.com/kb/infographics/11-tips-creating-infographics/)
% per effetti speciali (infografiche interattive), boh
% per scala, visual thinking (per layout) e visual doing
% per feed dati (praticamente da dove prendo i dati), boh (anche cap 9 libro infografiche)

%layout vedi foto visual leaders


%“Design della mente – infografica e data viz” di Paolo Bottazini, Michele Gotuzzo
%CAP 13: Estetica delle informazioni
Data viz è lo studio delle rappresentazioni grafiche dei dati, ovvero di come declinare le informazioni numeriche in modo tale da ottenere un'astrazione riassumendole in forma più schematica. 
Importante non troppo testo, usare infografica SOLO se abbastanza info da raccontare storia ma non troppe (che creano rumore).
%CAP 15: Gestalt (parte di design) 
Oltre a comunicazione visiva nel design è importante anche la gestalt che offre una serie di leggi che chiariscono come le persone categorizzano gli elementi che vedono e come sviluppano le loro idee a proposito degli eventi che gli accadono intorno. Leggi utili ai designer per organizzare meglio le informazioni. Queste leggi sono:
-	Di pregnanza, umani tendono a elaborare i pattern regolare e ordinati più velocemente di quelli complessi e articolati => organizzare dati logicamente in modo lineare e semplice
-	Di continuità, occhi istintivamente portati a raggruppare oggetti allineati => organizzare dati allineandoli per facilitare raggruppamento, comparazione e conseguente comprensione da parte del fruitore
-	Di similarità, oggetti che hanno caratteristiche simile (e.g. colore, forma, dim, orientamento) solitamente percepiti come un unico gruppo (rafforza dunque aggregazione, raggruppamento) => utilizzare caratteristiche simili per stabilire relazioni
-	Del punto focale, simmetrica alla precedente, in una rapp visuali oggetti diversi creano punti focali che attirano l'attenzione e dunque convincono l'osservatore delle differenze tra oggetti => utilizzare caratteristiche distintive per aumentare le differenze ed evidenziare ciò che interessa
-	Degli isomorfi, eventi e oggetti interpretati da utenti in base alla loro esperienza passata, dunque bisogna tener conto di convenzioni/condizionamenti culturali (e.g. rosso in occidente pericolo, perdita) => tenere sempre presenti le convenzioni e le abitudini preesistenti
-	Dell'immagine e dello sfondo, oggetti interpretati in maniera diversa in base alle relazioni con lo sfondo (se integrati perdono importanza, mentre separati hanno maggior rilievo) => assicurare un buon contrasto tra quelli che sono dati marginali come lo sfondo e le informazioni prioritarie da mettere in evidenza
-	Del fato o del destino comune, somiglianza di gruppo di ogg. consente di vederli come struttura (elem. che si spostano nella stessa direzione sono percepiti come maggiormente correlati => utilizzare le direzioni e i movimenti per stabilire o interrompere delle relazioni tra oggetti simili
%CAP 16: Il colore
Da usare con cura, in quanto il colore nell'infografica è strumento che permette di segnalare (e.g. dati specifici), sostenere, rafforzare o indicare il discorso narrativo. Inoltre, colore cattura l'attenzione, migliora l'informazione (facilitando anche la lettura, se ad alto contrasto, altrimenti poco leggibili se scelgo e.g. combinazioni rosso, marrone e verde), facilita il ricordo e influenza il comportamento del consumatore.
Ad ogni colore vengono associate sensazioni precise (che possono variare da area culturale/geografica qui passati dei valori generali, legati magari più all'occidente):
-	Colori caldi:
    o	Rosso:
        -	Uso: loghi, per attenzione, per att. promozionali, passione, velocità
        -	Valori: passione, amore, aggressività, emozioni forti, violenza
    o	Arancione
        -	Uso: fast food, banner pubblicità, richiamare attenzione e.g. per novità per temi giovanili e dinamici
        -	Valori: salute, energia, ottimismo, vitalità, appetito
    o	Giallo
        -	Uso: prodotti per bambini, prodotti innovativi, per richiamare attenzione
        -	Valori: vitalità, ottimismo, energia, allegria, felicità
        -	Colori freddi:
    o	Verde
        -	Uso: prodotti bio, eco e prodotti sanitari
        -	Valori: crescita, invidia, salute, natura, tranquillità
    o	Blu
        -	Uso: banche, compagnie assicurative, automobili
        -	Valori: affidabilità, onestà, sicurezza, calma, stabilità
    o	Porpora (viola)
        -	Uso: centri bellezza, beni di lusso, prodotti antietà
        -	Valori: eleganza, nobiltà, mistero, creatività, spiritualità
        -	Colori neutri
    o	Marrone
        -	Uso: prodotti naturali
        -	Valori: terra, affidabilità, sicurezza, tristezza, calore
    o	Bianco
        -	Uso: prodotti igienici legati a salute
        -	Valori: purezza, religiosità, matrimonio, pulizia, semplicità
    o	Nero
        -	Uso: prodotti lusso, attività proibite, negazione
        -	Valori: ribellione, eleganza, formalità, potenza, autorità
    o	Grigio
        -	Uso: prodotti tecnologici, moda
        -	Valori: sofisticato, neutralità, tristezza, moderno, raffinatezza
L'infografica deve mantenere un tono di colore univoco, possibilmente neutrale e sobrio.
Sensazioni di calma uso colori complementari, per qcsa di innovativo uso combinazioni stridenti.
%CAP 17: I caratteri
Importante nell'infografica (fondamentale in ogni attività che preveda rappr. grafica testuale), il loro utilizzo è contenuto in termini di quantità però è spesso la qualità a rendere incisivo un messaggio.
Regole basilari della composizione tipografica:
-	Graziato (serif), caratteri presentano piccole appendici che accompagnano linee ascendenti/discendenti lettera
o	Uso: per dare serietà al testo (hanno autorevolezza), usati per saggi, romanzi, quotidiani; su pc però meno leggibili
o	e.g. times new roman
-	Senza grazie (sans serif), senza le suddette appendici
o	Uso: ottimi per grandi titoli incisivi e per testi di larghezza contenuta; su schermi più leggibili dei suddetti. Nel caso delle infografiche è meglio usare questi.
o	E.g. Arial (meglio per testi più lunghi essendo più stretta), Verdana (meglio per poche frasi ad effetto), Helvetica, Rockwell
-	Carattere decorativo, come script calligrafici (difficili da utilizzare, da evitare su pc) e fumettistici (con curve morbidi e divertenti) etc. 
-	Chiaro (normale) e grassetto (bold) si riferiscono al peso del tratto (esistono anche varianti più leggere del chiaro tipo light, ultralight ma sono da evitare a schermo)
o	Uso: bold per componenti testuali brevi per focalizzare lo sguardo ed evidenziarli
-	Tondo (normale) e corsivo (italico) si riferisce all'inclinazione 
o	Uso: per citazioni e vocaboli particolari, non per grandi blocchi di testo in quanto poco leggibile
-	Maiuscolo corrisponde a “innalzamento della voce” da usare per omogeneizzare testi che risultano simili in termini di importanza e funzione (e.g. titoli) purché brevi (sennò difficilmente leggibile)
-	Sottolineatura nel web per collegamento ipertestuale, può essere usata per evidenziare ma grassetto è più efficace
-	Interlinea determina distanza tra riga e altra favorendo la leggibilità e se troppo compatta invece di difficile lettura
-	Spaziature e rientri
-	Allineamento e lunghezza, se centrato si presta per alcuni titoli meglio se non accompagnati da altro testo (con testi allineati in maniera diversa crea confusione); meglio preferire allineamenti a sinistra. Per quanto riguarda lunghezza testi troppo lunghi non vengono letti volentieri e sarebbe anche necessario usare caratteri più piccoli, meno leggibili
%CAP 18: Le immagini
Una visualizzazione più memorabile se:
-	contiene visualizzazioni che richiamano oggetti riconoscibili da umano
-	è facile distinguere elementi che la compongono
-	è colorata (almeno 5 colori e dati raggruppati)
-	è visivamente densa
-	possiede basso rapporto dati-inchiostro
Linguaggio immagini fortemente evocativo (valori simbolici). Alcune regole da usare per controllare espressività e costruire comunicazione efficace:
-	gestalt (vedi cap. 15)
-	lettura della pagina, dato dal diagramma di gutenberg (vedi foto) per cultura occidentale che suddivide l'area di lettura in 4 quadranti: punto di partenza in alto a sx (primaria importanza), in basso a dx dove termina la lettura, l'area in alto a dx e in basso a sx che sono di scarsa/debole lettura
-	immagini evocative, possono provocare un'esperienza sensorale
-	risoluzione dell'immagine

%intro: “L'arte funzionale – Infografica e visualizzazione delle informazioni” di Alberto Cairo
% cap 6
Capacità di prevedere il comportamento del cervello (i.e. meccanismi di percezione delle caratteristiche di base, dette pre-attentive) per avere migliori infografiche.
Il cervello riesce a percepire più velocemente variazioni di colore che di forma -> cervello ama contrasto/differenze. In generale, il cervello visivo è sostanzialmente un sistema che si è evoluto per individuare gli schemi (regioni nel campo visivo che condividono stessa natura o appartengono a entità diverse). Questi meccanismi di classificazione istantanea di differenze e somiglianze (che è percezione pre-attentiva) sono studiati nella psicologia della Gestalt, letteralmente forma, schema (Germania inizi xx secolo). Dice che il cervello non riconosce macchie di colore e forme come singole ma come complessi.
(vedi altri docs per singoli principi).
Scala cleveland e mcgill (vedi foto) da journal of the american statistical association, articolo “graphical perception:theory, experimentation, and application to the development of graphical methods), più si sale più accurate sono le valutazioni che i lettori riescono a fare in base ai grafici. Ci sono dieci attività percettive elementari, dove ciascuna costituisce un metodo per rappresentare i dati e classificati in base a come il cervello umano riesce a individuare le differenze e metterle a confronto (n.b. lunghezza, direzione e angolazione hanno uguale accuratezza; volume e curvatura pure; tonalità e intensità del colore anche).
Tutto questo tratta della percezione visiva di basso livello, i.e. differenziazioni tra primo piano e sfondo, stima dimensioni relative e deduzione semplici schemi ambiente.
Meccanismi di percezione delle caratteristiche di base, dette pre-attentive

% cap 9
“The design of everyday things” di Donald A. Norman che tratta di come ci rapportiamo a oggetti comuni, che sostiene la priorità dei bisogni dell'utente rispetto alle preoccupazioni estetiche dei designer. Principi proposti utili anche per infografiche (interattive):
-	Visibilità
o	Più visibile è la funzionalità di un oggetto, più facile sarà per gli utenti crearsi un modello mentale di ciò che possono ricavarne (evidenziare elem. importanti dell'infografica in modo che lettori possano percepirne pertinenza e funzionamento) -> norman descrive come “perceived affordances”, la forma di un oggetto deve suggerire visivamente cosa permette di fare (e.g. pulsante che assomigli a un pulsante reale)
o	Importante non solo per progettazione di interfacce ma anche per organizzazione degli elementi visivi, se un'info è indispensabile alla comprensione dell'intero resoconto dovrebbe essere sempre visibile e non nascosto sotto strato di interattività (utente non dovrebbe essere costretto a cliccare per visualizzare dati che dovrebbero essere sempre visibili)
-	Feedback
o	Per ogni azione, i lettori dovrebbero percepire una reazione, una risposta che indichi il buon esito dell'operazione che hanno cercato di compiere
-	Vincoli
o	Per evitare confusione, il designer deve porre intenzionalmente dei vincoli per orientare la navigazione dell'utente 
-	Uniformità
o	Entità di natura analoga dovrebbero somigliarsi
Principio delle informazioni visive di Ben Schneiderman “prima panoramica, zoom e filtri, poi dettagli su richiesta” che può essere ampliato a tutte le infografiche intende: 1 presentare i dati più importanti o punti più rilevanti e 2 permettere ai lettori di addentrarsi nelle info, esplorandone e dandone una propria lettura. Possono esserci infografiche lineari e non lineari ma comunque parte introduttiva con titolo e breve cappello. (vedi foto)
Jennifer Tidwell (designing interfaces) identifica diverse tecniche di esplorazione e navigazione delle infografiche:
-	Scorrimento e panning
o	E.g. scorrimento verticale su sito web, panning su mappa per spostarsi
-	Zoom
o	E.g. zoom su mappa
-	Apertura e chiusura
o	E.g. aprire e chiudere nuove finestre con dettaglio cosa 
-	Classificazione e riordino
o	E.g. riordino tabella
-	Ricerca e filtro
Queste possono però poi essere raggruppate/classificate in base a come gli utenti possono sperimentare le potenzialità dell'interfaccia (stili di interazione generali):
-	Istruzione
o	L'utente dice all'infografica di fare qualcosa (e.g. cliccando pulsanti)
-	Dialogo
o	L'utente dialoga con la presentazione (rara), cambiandone i parametri 
-	Manipolazione
o	L'utente cambia struttura o aspetto di ciò che viene presentato per raggiungere i loro obiettivi 
-	Esplorazione
o	E.g. muoversi all'interno di modelli 3D, sensazione di star dirigendo personalmente l'azione

%Design is storytelling” di Ellen Lupton
%atto 3
Vediamo il mondo in base a ciò che vogliamo fare, cerchiamo dei pattern con i nostri sensi e agiamo in base a queste percezioni. In base a quello visto in precedenza, proviamo a prevedere quello che accadrà (tendiamo a vedere solo quello che stiamo cercando)
Gestalt:
-	Prossimità, elementi posti uno vicino all'altro formano gruppi
-	Similarità, elementi di stessa forma o colore fanno parte di uno stesso gruppo
-	Fato comune, elementi sembrano cambiare come gruppo
-	Figure/ground ambiguity, gli spazi bianchi sono letti o come sfondo o come in primo piano
-	Chiusura e continuazione, mentalmente vengono chiusi gli spazi su linee o su forme regolari 
Un oggetto che innesca un'azione viene detto “affordance”, alcune di queste “affordance” sono accidentali (e.g. davanzale vicino a fermata del bus, perfetto per posare il caffè), altre sono imparate con il tempo (e.g. barre e bottoni nei website), magari rifacendosi ad oggetti reali

% (da vedere) su utils VIF... e narrative flow (?)

% “Visual thinking” di Willemien Brand 
Chiedersi quali linee sono necessarie per disegnare x (meno linee + impatto).
Quando si racconta una storia usiamo il linguaggio del corpo per stabilire un ordine/direzione, stessa cosa si può fare per i template di design. Si riportano di seguito
-	La lista, storia va da sopra v/sotto. Utile per agende, programmi e timetables
-	Vari steps, storia va da sotto v/alto. Utile per roadmaps (che va verso l'orizzonte in alto), calcoli
-	Timeline, storia va da dx v/sx. Utile per mostrare step cronologici, timelines (anche con diversi livelli), situazioni desiderabili e cambiamenti
-	Road, da basso sx a alto sx. Utile per rappresentare step da fare, un viaggio del consumatore, milestones da realizzare e possibile opportunità/pericoli lungo la strada per raggiungere obiettivo finale
-	Mandala, dal centro v/quattro direzioni diagonali. Utile per brainstorms e sessioni interattivi senza un risultato preciso.
-	Matrice, quattro angoli collegati tra loro. Utile per mostrare differenti tipi di in formazioni su un argomento (e.g. SWOT analysis) e per mostrare do e do not.

% “Visual doing” di Willemien Brand 
Non utilizzare più di sei livelli su cui sviluppare gli elementi:
-	Titolo principale grande (main topic)
-	Sottotitolo (main content)
-	Content e sub-content
-	Elementi di supporto (linee direzionali, divisori e contenitori, dettagli e side notes)
-	Segreti, dettagli e side notes (e tutto quello di cui non siè sicuri)
-	Highlights, triggers

% https://www.sciencedirect.com/book/9780128235676/visual-thinking-for-information-design
% cap 3 vedi immagini da file infografiche

%https://medium.com/@tetracubetech/infographics-purpose-elements-and-types-ae8bfc7cd89b
Elementi di una infografica:
-	Elementi visivi (grafiche, colori, icone)
-	Elementi conoscitivi (fatti)
-	Elementi di contenuto (statistiche, references etc.)


\section{Come costruire un'infografica}
% (da vedere) file synthesis... su utils e immagine content...(sempre su utils)

%Infographics: The New Communication Tools in Digital Age, Waralak V. Siricharoen
Come creare infografica
5 step:
1.	Get the idea
2.	Sketch it out, Bozza/prototipo dei component principali e come dovrebbero essere creati nell'infografica
3.	Collect the data/information
4.	Develop PoC
5.	Lay it out with styles, aggiungere tutto assieme

%“Design della mente – infografica e data viz” di Paolo Bottazini, Michele Gotuzzo 
%CAP 2: Regole della creatività (3 per chartjunk)
Infografica deve saper comunicare idee complesse con chiarezza, precisione ed efficienza. Per far ciò l'infografica deve (preso da edward tufte, visual display of quantitative information):
-	Mostrare i dati e renderne coerenti le varie tipologie (specie per grandi qtà)
    o	Scegliere set allestisce la “scena” del racconto
-	Indurre il lettore a pensare alla sostanza piuttosto che alla metodologia, la progetto grafico e simili
-	Evitare di distorcere quello che i dati devono dire
-	Incoraggiare il confronto tra classi di dati
-	Rivelare i dati a vari livelli di dettaglio
    o	Narrazione su più piani, diversa granularità, si approfondisce 
-	Seguire uno scopo chiaro, sia esplicativi che decorativi
    o	Trasparenza dei dati e solo nel secondo piano design, non esibire bravura di tecnica nello sviluppo di questi (chartjunk, no fare solo perché tecnologia permette, non decorare esposizione informazioni, e.g. effetto moirè (che avviene anche per barre equidistanti) o griglie inutili (specie se in nero scuro, meglio se grige se necessario) o tutti i comportamenti che fanno sì di dover compiere uno sforzo maggiore per la decifrazione del grafico)
-	Integrata con descrizioni statistiche e verbali dell'archivio dei dati
%CAP 6: Uno sguardo sul piano di lavoro (INFOMODEL)
InfoModel (vedi foto) comprende tutti i passaggi da eseguire per la presentazione di un lavoro infografico che abbia senso. È composto dai seguenti stanze (in cui il progetto si articola):
•	identificazione degli interlocutori, per cercare di rappresentare l'idea che queste persone vogliono
•	identificazione della storia e suoi effetti
•	identificazione dei dati disponibili
•	preparazione ed analisi dei dati, per comprendere le conseguenze tecniche (qualità, rappresentatività e compatibilità reciproca) delle scelte compiute nelle stanze precedenti
•	rappresentazione/narrazione della storia (metafora narrativa, i.e. applicare strutture narratologiche e retoriche per creare un discorso suggestivo)
%CAP 7: Identificare il tipo di pubblico (infomodel 1)
Due piani del discorso: logico (senso logico frasi) e pragmatico (persuasioni ed azioni che derivano dall'enunciazione delle frasi) detti anche da John Austin in locutorio e illocutorio (o performativo).
Persuasione = a volte più efficace di imporre la propria volontà, bisogna avvicinare pubblico intuendo quali sono sue inclinazioni cognitive (modalità di comprensione e di analisi dei fenomeni, categorizzate con il myers briggs type indicator) e decisionali (modalità apprendimento schematizzate da McCharthy).
(vedi foto)
Possibile costruire modello di classificazione decisionale per costruire racconti adeguati a ciascun interlocutore. E.g. mi chiedo se avrò di fronte qualcuno molto legato alla concretezza dell'esperienza, dei req. Etc. o qualcuno che vuole più percorrere un percorso concettuale. Poi ci si può chiedere se l'esposizione deve concentrarsi sulle prestazioni o sull'organizzazione sociale.
%CAP 8: Formulare la storia (infomodel 2)
Storia composta da azioni ed agenti
Struttura classifica del racconto (e.g. composta da 3 atti – equilibrio iniziale e sua rottura, ovvero inizio e movente/complicazione, peripezie del'eroe, ristabilimento dell'equilibrio, i.e. la conclusione - e dall'intervento di agenti che ricoprono il ruolo dell'eroe, del cattivo, dell'aiutante e del falso eroe) assicura la cattura di interesse dei lettori.
Di solito, infatti, si ha un paragone, inserito all'interno di un contesto storico (mostra stato iniziale e conclusivo).
%CAP 9: Individuare il dataset (infomodel 3)
Scegliere i dataset in base anche al pubblico scelto 
È innanzitutto necessario verificare il contenuto del dataset, verificando fonti e rappresentatività del campione e pulendolo da eventuali incoerenze di notazione (o per renderli coerenti al resto).
Dispositivi narrativi seguenti (che vengono innescati dai dataset):
•	proporzioni, per dimensionare il fenomeno confrontandolo con grandezze note o intuibilmente chiare (e.g. fatturato azienda con pil nazione)
•	comparazioni interne, mette in luce strategie, anche inconsapevoli, che governano il fenomeno sulla base dei pesi degli elementi (e.g. investimenti)
•	comparazioni esterne, confrontano strategie riferite a risorse interne rispetto all'uso consueto
•	cambiamenti nel corso del tempo
•	classifiche
•	analisi per categoria
•	associazioni 
%CAP 10: Raccontare la storia con i dati (infomodel 4 e 5)
Bisogna esaminare bene l'archivio di dati prima di rappresentarli, rappresentarli (viz preliminare e scriversi le prime impressioni che emergono alla viz, per capire pregiudizi. Poi interpretare meglio), trovare pattern e trasformarne poi l'impianto per esaltare le regolarità (e.g. con zoom, aggregazioni per isolare un gruppo, filtri o riduzione del rumore. Trasformazioni da fare con attenzione!).
Ricordandosi di indirizzare i dati al pubblico giusto.
%CAP 11: Rappresentare storie con i dati 
Guardare oltre ai dati, essi raccontano una storia di cui possono essere mostrati alcuni diversi aspetti in base a come i dati stessi vengono esposti e relazionati.
Che cosa cercare nelle relazioni?
Trovare relazioni con modelli in termini di tempo e proporzioni, poi cercare relazioni tra diverse variabili (c'è rapporto causa-effetto, c'è correlazione?) 
Aspetto visivo
Importante fare delle bozze di visualizzazione, cercnado di mediare le seguenti caratterstiche (più complessa/approfondita-più semplice/superficiale):
-	astrazione/raffigurazione
-	funzionalità/decorazione
-	densità/leggerezza (più dati o meno)
-	originalità/familiarità
-	novità/ridondanza (ripetere concetti in più modi)
-	multidimensionalità/unidimensionalità (contemporaneamente più aspetti o concentrati su pochi)
Il processo di trasformazione del dato
-	Complessità
-	Aggregazione
-	Selezione 
-	Formalizzazione 

%"L'arte funzionale – Infografica e visualizzazione delle informazioni” di Alberto Cairo
% cap 3
La complessità dell'infografica dovrebbe essere adatta alla natura del vostro lettore medio (per complessità di faccia riferimento alla ruota delle complessità, i.e. per mia infografica potrei provare a fare radar chart su questa ruota, in cui si hanno i seguenti assi:
-	Astrazione-raffigurazione, un'infografica è completamente figurativa (più raffigurazione) quando il rapporto tra il referente (soggetto da rappresentare) e la sua rappresentazione è perfettamente mimetica (i.e. tipo foto, mentre astrazione può essere ad esempio pittogramma)
-	Funzionalità-decorazione, inclusione o meno di elem. visivi che non vengono utilizzati direttamente per favorire la comprensione del materiale (non si intendono dunque font, colori che giocano un ruolo nella comprensibilità), i.e. elementi decorativi
-	Densità-leggerezza, quantità di dati presentati in relazione allo spazio utilizzato
-	Multidimensionalità-unidimensionalità, numero di livelli di approfondimento di un'infografica
-	Originalità-familiarità, forme grafiche comuni (e.g. pie, bar chart) o meno viste
-	Novità-ridondanza, spiegare le cose una volta sono o più volte con diversi mezzi (novità per non annoiare e ridondanza per far capire bene, quindi è importante trovare un equilibrio)
)
% cap 4
Come costruire l'infografica:
-	Pensare dapprima a assi densità-leggerezza e multidimensionalità-unidimensionalità
    o	Sostare di almeno un 10\% verso densità e multidimensionalità (si tende spesso a sottostimare capacità dei lettori)
    o	Organizzare l'infografica in livelli
    o	Fornire una sintesi dei dati (introduzione, media, dati salienti) che rappresenta il punto d'ingresso nell'infografica
    o	Sotto lo strato esterno, inserire il maggior numero di sottostrati (non tutta l'info ma solo quello che serve all'obiettivo)
    o	Sistemare gli strati in ordine logico
-	Pensare prima alla struttura e come organizzare i dati e poi ai fronzoli, penso all'asse funzionalità-decorazione (e astrazione-raffigurazione, non cercare di mettere solo immaginette perché si pensa che i lettori non ci arrivino altrimenti)
-	Penso agli assi originalità-ridondanza e novità-familiarità, sperimentare forme nuove è una necessità; tuttavia, meno comune è la forma grafica che scelgo per la mia visualizzazione più ridondanze dovrò inserire. In ogni caso è importante tenere conto dei lettori che potrebbero preferire diversi stili e ciò che voglio suscitare può essere diverso

%Design is storytelling” di Ellen Lupton
%atto i e ii
ATTO 1: pattern storie (pianificazione dello scenario)
Ingredienti di una storia:
-	Arco narrativo, la storia ha un inizio, una parte centrale e una fine
-	Cambiamento, l'azione trasforma il personaggio o la situazione
-	Tema, l'azione trasmette un significato
-	Coerenza, l'azione si deve basare su dettagli concreti e rilevanti
-	Plausibilità, l'azione è credibile e segue delle regole 
Arco narrativo/struttura della storia:
1.	Esposizione
2.	Crescita dell'azione
3.	Climax
4.	Decrescita dell'azione
5.	Epilogo
Storia include conflitto e suspence, la storia è il processo di risponde alla domanda iniziale e risolvere l'incertezza, se questa domanda viene risposta troppo in fretta la storia diventa banale/noiosa. 
Regola dei 3: struttura a 3 parti (max 4) per costruire storie (inizio, centro e fine) e interazioni sorprendenti e soddisfacenti (“Ready, set e go” mi dà l'impressione di un processo semplice), infatti l'ultimo elemento rompe lo schema dei primi 2)
ATTO 2: esprimere emozioni
Pensare a come utente possa sentirsi durante l'esperienza e come se la ricorderà poi, il designer deve empatizzare con i valori dell'utente, le aspirazioni, la cultura.
3 livelli di user experience (di Don Norman's)
-	Viscerale, all'inizio (tempo presente), il design provoca una reazione istantanea dovuta a forma, colore, texture e materiale
-	Comportamentale, al centro (tempo presente), il design provoca una reazione fisica o un'azione (e.g. cliccare bottone, comprare etc.)
-	Riflessiva, alla fine (tempo futuro), il design innesta memorie e associazioni relative al prodotto che rimangono nel tempo
Creare molteplici “persona”, ovvero archetipi di utenti in modo da capire come differenti persone con diversi desideri e abilità si approcciano/vivono lo strumento/servizio. Si concentrano sul perché fanno le cose queste “persone” più che sulla singola azione, ricostruendone backstory, emozioni, risorse, obiettivo e scenario.
Il colore è importante, crea una impressione sensoriale che riflette umore ed emozione. Un cambiamento nel “clima”/palette dei colori può essere utilizzato per un cambio di mood nel drama. Pantone 448 per disgusto


