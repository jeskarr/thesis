\chapter{Applicazione pratica di infografiche web con D3.js}
\label{cap:applicazione}
\intro{In questo capitolo verrà presentato il processo di sviluppo - comprendente progettazione, codifica e validazione - di un'infografica web sviluppata durante il corso dello stage.
Tale infografica ha lo scopo di presentare il corso di laurea triennale di Informatica dell'Università degli Studi di Padova.}\\

\section{Progettazione dell'infografica}
La scelta di realizzare un'infografica sul corso di laurea triennale in Informatica dell'Università degli Studi di Padova è stata motivata dall'esperienza personale dello stagista ome 
studente del corso, prossimo alla sua conclusione. 
Questo background ha consentito di trattare il tema con l'\emph{ethos} e l'autorevolezza necessari per creare un'infografica informativa e credibile.
Una solida conoscenza e competenza sull'argomento della presentazione sono, infatti, essenziali per garantire la qualità e l'accuratezza delle informazioni esposte.  

Inoltre, la disponibilità di dati pubblici e di fonti autorevoli, come i report dell'Università stessa e le statistiche di Almalaurea (il più importante consorzio interuniversitario italiano), ha 
garantito anche l'affidabilità del contenuto, contribuendo alla scelta di questo argomento.

\bigskip
\noindent Per quanto riguarda il processo di progettazione dell'infografica, questo ha seguito gli \emph{step} delineati dall'\emph{Infomodel}, come descritto nel capitolo precedente. 
Si riportano di seguito le varie scelte progettuali effettuate per ciascuna fase del processo.

\subsection{Identificazione degli interlocutori}

\subsection{Identificazione della storia e dei suoi effetti}

\subsection{Identificazione del dataset}

\subsection{Analisi e rappresentazione della storia}



\section{Codifica dell'infografica}
\subsection{Tecnologie e strumenti}\label{subsec:tecnologie}
(d3.js)
\subsection{Design Pattern adottati}
\subsection{Codifica}



\section{Verifica e validazione dell'infografica}
\subsection{Test implementati}
\subsection{Metodo di validazione}
validazione a/b o chatbot

