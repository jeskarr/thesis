\chapter{Introduzione}
\label{cap:introduzione}
\intro{In questo capitolo verranno delineate le aspettative iniziali e i piani per lo stage, 
inquadrando il tutto nel contesto aziendale.}

\section{Il contesto di riferimento}
\subsection{L'azienda ospitante}
Zucchetti S.p.A. nasce nel 1978 da un'intuizione di Domenico ``Mino'' Zucchetti, 
che per primo in Italia realizzò un software capace di realizzare la dichiarazione
dei redditi in maniera automatizzata.
Questo prodotto ebbe un grande successo e l'azienda iniziò quindi a produrre
software per automatizzare altre procedure contabili aziendali.

\begin{figure}[h!] 
    \centering 
    \includegraphics[width=0.6\columnwidth]{zucchetti-logo.png} 
    \caption{Logo Zucchetti S.p.A.}
\end{figure}

\noindent A oggi, Zucchetti è una delle principali aziende in Italia nel settore ICT 
(\emph{Information and Communication Technology}) e un attore sempre 
più rilevante sul mercato internazionale.
Il gruppo può infatti contare più di 9.000 dipendenti distribuiti in 15 paesi 
e oltre 700.000 clienti presenti in più di 50 nazioni.

Oltre al consolidamento della sua presenza sul territorio, Zucchetti ha anche 
ampliato la sua offerta di prodotti.
Essa fornisce gestionali aziendali e soluzioni ERP (\emph{Enterprise Resource Planning})
per la gestione del personale, le comunicazione aziendali, la rilevazione delle 
presenze e degli accessi, etc.. 
Zucchetti opera, inoltre, nel campo dell'e-business e della business intelligence e
offre servizi per robotica, automazione, sicurezza informatica e server farm.

\subsubsection{Servizi offerti}
La \emph{mission} di Zucchetti è quella di supportare le aziende nel migliorare
la loro competitività ed efficienza operativa attraverso soluzioni innovative e 
di qualità.
A tal fine, Zucchetti S.p.A. fornisce ai suoi clienti:
\begin{itemize}
    \item \textbf{Un supporto pre-vendita}, analizzando e studiando le soluzioni 
    che meglio si adattano al problema;
    \item \textbf{Un supporto post-vendita}, installando il prodotto e 
    offrendo supporto tecnico;
    \item \textbf{Una formazione del personale}, per consentire al cliente di 
    sfruttare al meglio il prodotto acquistato;
    \item \textbf{Un aggiornamento costante}, per restare al passo con le nuove
    tecnologie e le nuove normative in materia fiscale, contabile e amministrativa.
\end{itemize} 

\subsubsection{Clientela target}
Zucchetti S.p.A. si rivolge a una vasta gamma di clientela, comprendente imprese
di ogni dimensione e settore, ma anche professionisti e associazioni di categoria, 
oltre che CAF e Pubblica Amministrazione. 
La sua offerta diversificata di soluzioni software e servizi tecnologici è pensata 
infatti per soddisfare le esigenze specifiche di ciascun segmento di mercato.


\section{Aspettative iniziali}
\subsection{La proposta di stage}
Lo scopo del progetto di stage è quello di sviluppare un sistema di visualizzazione interattiva dei dati che 
fornisca agli utenti una comprensione non convenzionale, ma pur sempre intuitiva, delle analisi effettuate. 
Per realizzare tale visualizzazione viene proposto l'utilizzo della libreria JavaScript ``D3.js'', nota per la 
sua flessibilità e capacità di creare visualizzazioni grafiche dinamiche e coinvolgenti.

A tal fine, è necessario studiare e analizzare in profondità le varie modalità di rappresentazione
grafica dei dati, focalizzandosi sull'identificazione delle tecniche più idonee per rendere chiari e
informativi i vari tipi di dati possibili. A partire da questo studio, si potrà passare alla creazione di un'infografica
web, che combinerà grafici realizzati tramite ``D3.js'' con altri elementi aggiuntivi di chiarimento. Una
volta completata l'infografica, si prevede di sviluppare un template riutilizzabile per future rappresentazioni
simili e verificarne l'efficacia e l'utilità mediante adeguati metodi di valutazione. 

Questo progetto di stage offre dunque l'opportunità di approfondire la propria conoscenza dei seguenti campi:
\begin{itemize}
    \item \textbf{Data Visualization}, attraverso lo studio di infografiche e, in particolare, di grafici informativi;
    \item \textbf{Utilizzo della libreria D3.js}, attraverso la creazione di grafici dinamici e interattivi inseriti all'interno dell'infografica;
    \item \textbf{Sviluppo Web con HTML/CSS/JS}, attraverso la realizzazione di un'interfaccia web che implementi l'infografica.
\end{itemize}

\subsection{Pianificazione delle attività}
Si riporta di seguito la pianificazione delle attività di stage suddivise 
nelle otto settimane di lavoro previste.

\begin{itemize}
    \item \textbf{Prima Settimana - Formazione}
    \begin{itemize}
        \item Introduzione e approfondimento sui principi chiave e sulle metodologie della \textit{Data Visualization};
        \item Formazione sulle tecnologie da adottare, in particolare sulla libreria ``D3.js''.
    \end{itemize}
    \item \textbf{Seconda Settimana - Selezione e sviluppo di grafici} 
    \begin{itemize}
        \item Identificazione dei criteri per la scelta del tipo di grafico più adatto, considerando la struttura dei dati e gli obiettivi della visualizzazione;
        \item Sviluppo dei grafici individuati utilizzando la libreria ``D3.js''.
    \end{itemize}
    \item \textbf{Terza Settimana - Integrazione di algoritmi a grafici} 
    \begin{itemize}
        \item Sviluppo di grafici avanzati, che integrano algoritmi, attraverso l'uso di ``D3.js''.
    \end{itemize}
    \item \textbf{Quarta Settimana - Test e Documentazione} 
    \begin{itemize}
        \item Realizzazione di test che verifichino la bontà del codice prodotto nelle settimane precedenti;
        \item Produzione di documentazione riguardante il prodotto realizzato.
    \end{itemize}
    \item \textbf{Quinta Settimana - Approfondimento sulle infografiche} 
    \begin{itemize}
        \item Ricerca approfondita di informazioni e guide riguardanti il concetto di infografica;
        \item Raccolta di elementi grafici e di design da accompagnare ai grafici per migliorarne la comprensibilità e l'attrattività visiva;
        \item Realizzazione preliminare dell'infografica utilizzando HTML, CSS e JavaScript.
    \end{itemize}
    \item \textbf{Sesta Settimana - Completamento dell'infografica e crezione del template} 
    \begin{itemize}
        \item Ultimazione dell'infografica in HTML, CSS e JavaScript;
        \item Creazione di un template dell'infografica, riutilizzabile per future analisi simili.
    \end{itemize}
    \item \textbf{Settima Settimana - Metodi di validazione dell'infografica} 
    \begin{itemize}
        \item Individuazione di criteri che permettano di valutare la qualità dell'infografica (e.g. chatbot con LLM che interagisce con l'utente sull'infografica ottenendo, indirettamente, il suo feedback per miglioramenti futuri);
        \item Implementazione dei metodi di validazione individuati.
    \end{itemize}
    \item \textbf{Ottava Settimana - Test e Documentazione} 
    \begin{itemize}
        \item Realizzazione di test che verifichino la bontà del codice prodotto nelle settimane precedenti;
        \item Produzione di documentazione riguardante il prodotto realizzato.
    \end{itemize}
\end{itemize}

\subsection{Obiettivi fissati}
Sono fissati i seguenti obiettivi da raggiungere attraverso le attività sopraelencate.

\subsubsection{Obiettivi obbligatori}
\begin{itemize}
    \item \underline{\textit{O01}}: Comprendere i principi fondamentali della \textit{Data Visualization};
    \item \underline{\textit{O02}}: Selezionare il grafico più adatto per tipo di struttura dati e obiettivo della visualizzazione;
    \item \underline{\textit{O03}}: Realizzare grafici informativi e interattivi utilizzando ``D3.js'';
    \item \underline{\textit{O04}}: Realizzare un'infografica web funzionante in HTML/CSS/JS che visualizzi i risultati degli algoritmi di Intelligenza Artificiale e Machine Learning;
    \item \underline{\textit{O05}}: Creare un template dell'infografica;
    \item \underline{\textit{O06}}: Individuare dei metodi di validazione dell'infografica.
\end{itemize}

\subsubsection{Obiettivi desiderabili}
\begin{itemize}
    \item \underline{\textit{D01}}: Sviluppare infografiche alternative;
    \item \underline{\textit{D02}}: Implementare un chatbot con LLM che interagisca con l'utente sull'infografica.
\end{itemize}

\subsection{Vincoli}
Si riportano di seguito i vincoli imposti per lo svolgimento dell'attività di stage.

\subsubsection{Vincoli temporali}
La durata dello stage è di otto settimane, per un totale complessivo compreso tra le 300 e le 320 ore in base alla nacessità.
Gli orari di lavoro sono dal lunedì al venerdì, dalle ore 9.00 alle ore 18:00, con pausa pranzo dalle 13:00 alle 14:00.

\subsubsection{Vincoli metodologici}
In comune accordo con il tutor aziendale, l'attività lavorativa è da svolgersi in presenza.

\subsubsection{Vincoli tecnologici}
La proposta di stage prevede l'utilizzo della libreria ``D3.js'' per lo sviluppo
di grafici interattivi. Questi andranno poi ad inserirsi in infografiche web da 
sviluppare attraverso HTML, CSS e JavaScript.
Non sono stati imposti limiti ulteriori su eventuali tecnologie aggiuntive.

Per una specifica delle tecnologie effettuate si rimanda alla sezione apposita \ref{subsec:tecnologie}.

% motivazioni personali ?

%analisi preventiva rischi
%Durante la fase di analisi iniziale sono stati individuati alcuni possibili rischi a cui si potrà andare incontro.
%Si è quindi proceduto a elaborare delle possibili soluzioni per far fronte a tali rischi.\\

%\begin{risk}{Performance del simulatore hardware}
 %   \riskdescription{le performance del simulatore hardware e la comunicazione con questo potrebbero risultare lenti o non abbastanza buoni da causare il fallimento dei test}
  %  \risksolution{coinvolgimento del responsabile a capo del progetto relativo il simulatore hardware}
   % \label{risk:hardware-simulator} 
%\end{risk}