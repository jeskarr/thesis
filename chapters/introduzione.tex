\chapter{Introduzione}
\label{cap:introduzione}

\section{Il contesto di riferimento}
\subsection{L'azienda ospitante}
Zucchetti S.p.A. nasce nel 1978 da un'intuizione di Domenico ``Mino'' Zucchetti, 
che per primo in Italia realizzò un software capace di realizzare la dichiarazione
dei redditi in maniera automatizzata.
Questo prodotto ebbe un grande successo e l'azienda iniziò quindi a produrre
software per automatizzare altre procedure contabili aziendali.

\begin{figure}[h!] 
    \centering 
    \includegraphics[width=0.6\columnwidth]{zucchetti-logo.png} 
    \caption{Logo Zucchetti S.p.A.}
\end{figure}

\noindent A oggi, Zucchetti è una delle principali aziende in Italia nel settore ICT 
(\emph{Information and Communication Technology}) e un attore sempre 
più rilevante sul mercato internazionale.
Il gruppo può infatti contare più di 9.000 dipendenti distribuiti in 15 paesi 
e oltre 700.000 clienti presenti in più di 50 nazioni.

Oltre al consolidamento della sua presenza sul territorio, Zucchetti ha anche 
ampliato la sua offerta di prodotti.
Essa fornisce gestionali aziendali e soluzioni ERP (\emph{Enterprise Resource Planning})
per la gestione del personale, le comunicazione aziendali, la rilevazione delle 
presenze e degli accessi, etc.. 
Zucchetti opera, inoltre, nel campo dell'e-business e della business intelligence e
offre servizi per robotica, automazione, sicurezza informatica e server farm.



\subsubsection{Servizi offerti}
La \emph{mission} di Zucchetti è quella di supportare le aziende nel migliorare
la loro competitività ed efficienza operativa attraverso soluzioni innovative e 
di qualità.
A tal fine, Zucchetti S.p.A. fornisce ai suoi clienti:
\begin{itemize}
    \item \textbf{Un supporto pre-vendita}, analizzando e studiando le soluzioni 
    che meglio si adattano al problema;
    \item \textbf{Un supporto post-vendita}, installando il prodotto e 
    offrendo supporto tecnico;
    \item \textbf{Una formazione del personale}, per consentire al cliente di 
    sfruttare al meglio il prodotto acquistato;
    \item \textbf{Un aggiornamento costante}, per restare al passo con le nuove
    tecnologie e le nuove normative in materia fiscale, contabile e amministrativa.
\end{itemize} 

\subsubsection{Clientela target}
Zucchetti S.p.A. si rivolge a una vasta gamma di clientela, comprendente imprese
di ogni dimensione e settore, ma anche professionisti e associazioni di categoria, 
oltre che CAF e Pubblica Amministrazione. 
La sua offerta diversificata di soluzioni software e servizi tecnologici è pensata 
infatti per soddisfare le esigenze specifiche di ciascun segmento di mercato.


\section{Aspettative iniziali}
\subsection{La proposta di stage}

\subsection{Piano di lavoro}

\subsection{Vincoli e obiettivi}

\subsection{Motivazioni personali}


analisi preventiva rischi
Introduzione all'idea dello stage, preventivo ore e requisiti, obiettivi da raggiungere.
rischi.
Durante la fase di analisi iniziale sono stati individuati alcuni possibili rischi a cui si potrà andare incontro.
Si è quindi proceduto a elaborare delle possibili soluzioni per far fronte a tali rischi.\\

\begin{risk}{Performance del simulatore hardware}
    \riskdescription{le performance del simulatore hardware e la comunicazione con questo potrebbero risultare lenti o non abbastanza buoni da causare il fallimento dei test}
    \risksolution{coinvolgimento del responsabile a capo del progetto relativo il simulatore hardware}
    \label{risk:hardware-simulator} 
\end{risk}


\noindent Esempio di utilizzo di un termine nel glossario \\
\gls{api}. \\

\noindent Esempio di citazione in linea \\
\cite{site:agile-manifesto}. \\

\noindent Esempio di citazione nel pie' di pagina \\
citazione\footcite{womak:lean-thinking} \\