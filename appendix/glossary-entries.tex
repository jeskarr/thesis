% Acronyms
\newacronym[description={\glslink{llmg}{Large Language Model}}]
    {llm}{LLM}{Large Language Model}

\newacronym[description={\glslink{ictg}{Information and Communication Technology}}]
    {ict}{ICT}{Information and Communication Technology}

\newacronym[description={\glslink{erpg}{Enterprise Resource Planning}}]
    {erp}{ERP}{Enterprise Resource Planning}

\newacronym[description={\glslink{htmlg}{Hypertext Markup Language}}]
    {html}{HTML}{Hypertext Markup Language}

\newacronym[description={\glslink{cssg}{Cascading Style Sheets}}]
    {css}{CSS}{Cascading Style Sheets}

\newacronym[description={\glslink{jsg}{JavaScript}}]
    {js}{JS}{JavaScript}

\newacronym[description={\glslink{iclg}{In-Context Learning}}]
    {icl}{ICL}{In-Context Learning}

\newacronym[description={\glslink{fslg}{Few-Shot Learning}}]
    {fsl}{FSL}{Few-Shot Learning}

\newacronym[description={\glslink{vifg}{Visual Information Flow}}]
    {vif}{VIF}{Visual Information Flow}

\newacronym[description={\glslink{mbtig}{Myers-Briggs Type Indicator}}]
    {mbti}{MBTI}{Myers-Briggs Type Indicator}
    
\newacronym[description={\glslink{wcagg}{Web Content Accessibility Guidelines}}]
    {wcag}{WCAG}{Web Content Accessibility Guidelines}
    
\newacronym[description={\glslink{svgg}{Scalable Vector Graphics}}]
    {svg}{SVG}{Scalable Vector Graphics}
    
\newacronym[description={\glslink{sqlg}{Structured Query Language}}]
    {sql}{SQL}{Structured Query Language}

\newacronym[description={\glslink{csvg}{Comma-Separated Values}}]
    {csv}{CSV}{Comma-Separated Values}

\newacronym[description={\glslink{rrfg}{Reciprocal Rank Fusion}}]
    {rrf}{RRF}{Reciprocal Rank Fusion}

\newacronym[description={\glslink{dbsfg}{Distribution-Based Score Fusion}}]
    {dbsf}{DBSF}{Distribution-Based Score Fusion}
    
% Glossary entries
\newglossaryentry{datavizg} {
    name=Data Visualization,
    text=Data Visualization,
    sort=datavisualization,
    description=
    {Processo di rappresentazione visiva dei dati attraverso grafici, tabelle e altre forme di visualizzazione grafica.
    Viene utilizzato per facilitare l'interpretazione e l'analisi delle informazioni, permettendo di identificare tendenze, \emph{pattern} e \emph{insight} 
    in modo chiaro e intuitivo, migliorandone la comprensione}
}

\newglossaryentry{d3g} {
    name=D3.js,
    text=D3.js,
    sort=d3,
    description=
    {Libreria \gls{js} \emph{open source} usata per la creazione, a partire da dati organizzati, di visualizzazioni interattive che siano visibili attraverso un comune browser.
    Il suo nome, infatti, sta per \emph{Data-Driven Documents} (documenti basati sui dati). 
    In particolare, consente di creare grafici interattivi e dinamici sul web, utilizzando \gls{html}, \gls{svg} e \gls{css}}
}

\newglossaryentry{ictg} {
    name=\glslink{ict}{ICT},
    text=Information and Communication Technology,
    sort=ict,
    description=
    {Tecnologie riguardanti i sistemi integrati di telecomunicazione, i computer, le tecnologie audio-video e relativi \emph{software}.
    In generale, dunque, tutte le tecnologie che permettono agli utenti di creare, immagazzinare e scambiare informazioni}
}

\newglossaryentry{erpg} {
    name=\glslink{erp}{ERP},
    text=Enterprise Resource Planning,
    sort=erp,
    description=
    {\emph{Software} utilizzati per la gestione quotidiana di imprese.
    Essi integrano tutti i processi e funzioni aziendali rilevanti, come ad esempio vendite, acquisti, gestione magazzino, finanza o contabilità}
}

\newglossaryentry{ebusinessg} {
    name=e-business,
    text=e-business,
    sort=ebusiness,
    description=
    {Termine generale che comprende tutte le attività commerciali che possono svolgersi online, ossia attraverso 
    Internet e altre reti telematiche. Comprende anche l'\emph{e-commerce}}
}

\newglossaryentry{businessintg} {
    name=business intelligence,
    text=business intelligence,
    sort=businessintelligence,
    description=
    {Tecnologie e processi aziendali volti a raccogliere ed analizzare i dati al fine di prendere decisioni più informate e 
    sviluppare strategie aziendali}
}

\newglossaryentry{serverfarmg} {
    name=server farm,
    text=server farm,
    sort=serverfarm,
    description=
    {Insieme di server collocati in un unico ambiente fisico.
    Vengono utilizzate per poter centralizzare la gestione, la manutenzione e la sicurezza dei server stessi.}
}

\newglossaryentry{mlg} {
    name=Machine Learning,
    text=Machine Learning,
    sort=machinelearning,
    description=
    {Branca dell'Intelligenza Artificiale che permette ai sistemi
    di apprendere da grandi set di dati e migliorare automaticamente dalle esperienze, senza ulteriore programmazione. 
    Utilizzano algoritmi statistici e modelli matematici per identificare schemi da queste grandi quantità di dati e fare
    previsioni basate su nuovi input}
}

\newglossaryentry{htmlg} {
    name=\glslink{html}{HTML},
    text=Hypertext Markup Language,
    sort=html,
    description=
    {Linguaggio di \emph{markup} ampiamente usato per i documenti web.
    Stabilisce tramite dei tag cosa un browser web deve mostrare a un visitatore al suo arrivo in un sito}
}

\newglossaryentry{cssg} {
    name=\glslink{css}{CSS},
    text=Cascading Style Sheets,
    sort=css,
    description=
    {Linguaggio usato per definire la formattazione di documenti scritti in un linguaggio di \emph{markup}, come ad esempio \gls{html}.
    Nello specifico, vengono definite delle regole che dicono al browser come gli elementi della pagina devono essere visualizzati}
}

\newglossaryentry{jsg} {
    name=\glslink{js}{JS},
    text=JavaScript,
    sort=js,
    description=
    {Linguaggio di programmazione ad alto livello interpretato e orientato agli eventi. Viene comunemente utilizzato per aggiungere interattività e dinamicità alle pagine web, 
    ma può essere utilizzato anche per lo sviluppo di applicazioni web, mobile e desktop}
}

\newglossaryentry{llmg} {
    name=\glslink{llm}{LLM},
    text=Large Language Model,
    sort=llm,
    description=
    {Tipologia di modello di AI progettato per comprendere e generare testo con una capacità simile a quella umana. 
    Utilizza reti neurali profonde per apprendere da grandi quantità di testi regole linguistiche, coerenza del contesto e stili di scrittura}
}

\newglossaryentry{infoloadg} {
    name=information overload,
    text=information overload,
    sort=informationoverload,
    description=
    {Anche detta ``sovraccarico informativo'', indica una serie di problemi psicologici, produttivi e decisionali che si verificano a causa di un'esposizione quotidiana 
    ad un numero troppo elevato di informazioni e stimolazioni.
    Si concretizza negli individui con una difficoltà nell'attenzione e nella comprensione di argomenti, nonché nella presa di decisioni}
}

\newglossaryentry{bigdatag} {
    name=big data,
    text=big data,
    sort=bigdata,
    description=
    {Grande quantità di dati informatici, estesa sia in termini di volume sia di varietà sia per velocità con cui vengono forniti. 
    Richiedono tecnologie e metodi analitici specifici per poterne estrarre informazioni utili}
}

\newglossaryentry{datascienceg} {
    name=Data Science,
    text=Data Science,
    sort=datascience,
    description=
    {Disciplina scientifica che combina competenze matematiche, statistiche, informatiche e di visualizzazione al fine di interpretare ed estrarre conoscenza dai dati}
}

\newglossaryentry{chartjunkg} {
    name=chartjunk,
    text=chartjunk,
    sort=chartjunk,
    description=
    {Termine coniato dallo statistico Edward R. Tufte per indicare gli elementi presenti in una visualizzazione che hanno solamente funzione decorativa o, più in generale, tutti gli elementi 
    che possono essere omessi senza perdere informazioni}
}

\newglossaryentry{sistemadiregoleg} {
    name=sistema di regole,
    text=sistema di regole,
    sort=sistemadiregole,
    description=
    {Sistema esperto che consente di compiere in automatico certe azioni prestabilite, evitando così l'intervento manuale di figure esperte.
    Partendo da una serie di dati in input, mette in atto autonomamente procedure di inferenza, di logica, giungendo così alla risoluzione di problemi complessi}
}

\newglossaryentry{promptengg} {
    name=Prompt Engineering,
    text=Prompt Engineering,
    sort=promptengineering,
    description=
    {Processo di progettazione e ottimizzazione delle istruzioni fornite a un modello di Intelligenza Artificiale, volto ad ottenere risposte più accurate e pertinenti
    da questi modelli. La qualità delle risposte, infatti, può variare di molto a seconda della formulazione del \emph{prompt}.}
}

\newglossaryentry{fslg} {
    name=\glslink{fsl}{FSL},
    text=Few-Shot Learning,
    sort=fsl,
    description=
    {Strategia di apprendimento automatico che utilizza un numero molto limitato di esempi di addestramento.
    Si sfruttano le informazioni apprese precedentemente da compiti o grandi set di dati per aiutare nella classificazione di nuove classi per le 
    quali vengono forniti solo alcuni esempi ciascuna.
    È un sottoinsieme del \emph{n-shot learning}}
}

\newglossaryentry{iclg} {
    name=\glslink{icl}{ICL},
    text=In-Context Learning,
    sort=icl,
    description=
    {Tecnica di \gls{promptengg} che consiste nell'inserire direttamente nel \emph{prompt} una dimostrazione del compito o esempio specifico di risposta che il modello 
    di Intelligenza Artificiale deve compiere/restituire}
}

\newglossaryentry{llamacppg} {
    name=llama.cpp,
    text=llama.cpp,
    sort=llamacpp,
    description=
    {Strumento \emph{open-source} progettato per eseguire modelli di Intelligenza Artificiale direttamente in C/C++.
    Consente anche di ricevere in risposta dal modello un output strutturato}
}

\newglossaryentry{jsonrulesg} {
    name=json-rules-engine,
    text=json-rules-engine,
    sort=jsonrulesengine,
    description=
    {Libreria \gls{js} che permette di definire e gestire regole basate su \gls{jsong} per la costruzione di un \gls{sistemadiregoleg}.
    Consente, dunque, di implementare logiche di business complesse attraverso un formato di regole dichiarative, definite in maniera semplice e leggibile}
}

\newglossaryentry{jsong} {
    name=JSON,
    text=JSON,
    sort=json,
    description=
    {Formato di scrittura (basato su \gls{js}) che si caratterizza per la sua grande leggibilità, dovuta ad una sintassi semplice e facilmente interpretabile dai computer. 
    Viene utilizzato in diversi contesti, tra cui lo sviluppo web e lo scambio di dati tra applicazioni}
}

\newglossaryentry{expdataanalysisg} {
    name=Exploratory Data Analysis,
    text=Exploratory Data Analysis,
    sort=exploratorydataanalysis,
    description=
    {Tecnica del campo della \gls{datascienceg}, usata per analizzare e indagare i set di dati e riassumerne le caratteristiche principali.
    Spesso impiega metodi di \gls{datavizg}}
}

\newglossaryentry{grammaticag} {
    name=grammatica,
    text=grammatica,
    sort=grammatica,
    description=
    {Nel presente documento, termine utilizzato nel senso di ``grammatica formale'' e indica un insieme di regole e sintassi formali utilizzate per 
    definire e vincolare il tipo di output che un modello di linguaggio può generare}
}

\newglossaryentry{vifg} {
    name=\glslink{vif}{VIF},
    text=Visual Information Flow,
    sort=vif,
    description=
    {Modo in cui gli elementi grafici vengono organizzati e collegati tra loro per guidare l'utente attraverso una storia visiva. 
    Tale struttura consente di comprendere e interpretare le informazioni con maggior chiarezza e coerenza}
}


\newglossaryentry{mbtig} {
    name=\glslink{mbti}{MBTI},
    text=Myers-Briggs Type Indicator,
    sort=mbti,
    description=
    {Indicatore psicologico che identifica, tramite questionari, quattro caratteristiche psicologiche che, a seconda del modo in cui queste si presentano, 
    determinano il modo in cui una persona si rapporta con il mondo e la vita in generale.
    La valutazione di tali caratteristiche porta all'identificazione di un tipo di personalità (tra i 16 disponibili, determinati da tutte 
    le possibili combinazioni di queste caratteristiche)}
}

\newglossaryentry{wcagg} {
    name=\glslink{wcag}{WCAG},
    text=Web Content Accessibility Guidelines,
    sort=wcag,
    description=
    {Parte di una serie di linee guida per l'accessibilità dei siti Web, pubblicate dal \emph{World Wide Web Consortium} (W3C)}
}

\newglossaryentry{svgg} {
    name=\glslink{svg}{SVG},
    text=Scalable Vector Graphics,
    sort=svg,
    description=
        {Formato di file utilizzato per descrivere grafica vettoriale bidimensionale. 
        In particolare, consente di creare immagini scalabili senza perdita di qualità, adatte per essere visualizzate su 
        dispositivi di diversi formati e dimensioni}
}

\newglossaryentry{pythong} {
    name=Python,
    text=Python,
    sort=python,
    description=
    {Linguaggio di programmazione ad alto livello, molto versatile e ampliamente utilizzato.
    Supporta diversi paradigmi di programmazione, come quello \emph{object-oriented}, quello imperativo e quello funzionale.
    Inoltre, offre una tipizzazione dinamica forte e presenta una sintassi molto chiara e leggibile}
}

\newglossaryentry{sqlg} {
    name=\glslink{sql}{SQL},
    text=Structured Query Language,
    sort=sql,
    description=
    {Linguaggio standardizzato usato per \emph{database} basati sul modello relazionale.
    Consente la creazione e la gestione delle strutture di \emph{database}, come tabelle e indici. Nello specifico, consente di eseguire anche operazioni di 
    interrogazione, aggiornamento, inserimento e cancellazione dei dati nelle tabelle}
}

\newglossaryentry{embeddingsg} {
    name=embedding,
    text=embedding,
    sort=embedding,
    description=
    {Rappresentazione di oggetti (come parole o frasi) in uno spazio vettoriale continuo. Queste rappresentazioni vettoriali catturano le relazioni e le somiglianze semantiche tra gli oggetti.
    Tali strutture sono fondamentali per compiti come la classificazione del testo, la ricerca per similarità e i sistemi di raccomandazione}
}

\newglossaryentry{crossoriging} {
    name=cross-origin,
    text=cross-origin,
    sort=crossorigin,
    description=
    {Termine che si riferisce alla politica di sicurezza dei browser web che gestisce le richieste tra domini differenti. 
    Nello specifico quando un'applicazione web fa una richiesta a un dominio diverso da quello da cui è stata originariamente caricata, si ha una richiesta \emph{cross-origin}.}
}

\newglossaryentry{geojsong} {
    name=GeoJSON,
    text=GeoJSON,
    sort=geojson,
    description=
    {Formato basato su \gls{jsong}, progettato per rappresentare dati geospaziali.
    È ampiamente utilizzato per archiviare e scambiare informazioni geografiche e geometrie spaziali}
}

\newglossaryentry{csvg} {
    name=\glslink{csv}{CSV},
    text=Comma-Separated Values,
    sort=csv,
    description=
    {Formato di file basato su file di testo, solitamente utilizzato per l'importazione ed esportazione di dati tabulari.
    Ogni riga della tabella viene rappresentata da una riga di dati nel file \emph{csv} e ogni campo in quella riga è separato da una virgola}
}

\newglossaryentry{stopwordsg} {
    name=stop-words,
    text=stop-words,
    sort=stopwords,
    description=
        {Parole comuni, come ad esempio articoli o congiunzioni, 
        che non si riferiscono ad un argomento specifico e dunque non contribuiscono al significato del testo.
        In operazioni di confronto tra testi, tali parole sono spesso rimosse per migliorare l'efficacia di queste operazioni}
}

\newglossaryentry{stemmingg} {
    name=stemming,
    text=stemming,
    sort=stemming,
    description=
        {Processo che riduce le parole alla loro radice o forma base.
        In operazioni di confronto tra testi, questo è utile in quanto permette di accorpare varianti di una parola alla stessa radice, 
        migliorando la precisione in tali confronti}
}

\newglossaryentry{rrfg} {
    name=\glslink{rrf}{RRF},
    text=Reciprocal Rank Fusion,
    sort=rrf,
    description=
        {Algoritmo utilizzato per combinare i risultati provenienti da più sistemi di ricerca. Ad esempio, può essere utilizzato per combinare la ricerca semantica con metodi statistici, realizzando così una ricerca ibrida. 
        Esso utilizza i \emph{rank} dei documenti ottenuti dai diversi sistemi e li aggrega in una classifica unificata, migliorando così la qualità dei risultati di ricerca complessivi}
}

\newglossaryentry{dbsfg} {
    name=\glslink{dbsf}{DBSF},
    text=Distribution-Based Score Fusion,
    sort=dbsf,
    description=
        {Algoritmo utilizzato per combinare i risultati provenienti da più sistemi di ricerca. Ad esempio, può essere utilizzato per combinare la ricerca semantica con metodi statistici, realizzando così una ricerca ibrida. 
        Esso utilizza i punteggi ottenuti dai diversi sistemi e li normalizza, indipendentemente dalla media della loro distribuzione, ottenendo così una classifica priva di \emph{bias} di normalizzazione}
}

\newglossaryentry{bm25g} {
    name=Ranking BM25,
    text=Ranking BM25,
    sort=rankingbm25,
    description=
        {Algoritmo di \emph{ranking} per il reperimento dell'informazione.
        Esso identifica la pertinenza di un documento rispetto a una certa \emph{query} e ordina i documenti sulla base dei loro punteggi di rilevanza. Tali punteggi sono determinati tenendo conto della frequenza con cui il termine 
        di ricerca compare nel documento, della lunghezza del documento e della lunghezza media di tutti i documenti disponibili}
}