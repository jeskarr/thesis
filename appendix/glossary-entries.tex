% Acronyms
\newacronym[description={\glslink{llmg}{Large Language Model}}]
    {llm}{LLM}{Large Language Model}

\newacronym[description={\glslink{ictg}{Information and Communication Technology}}]
    {ict}{ICT}{Information and Communication Technology}

\newacronym[description={\glslink{erpg}{Enterprise Resource Planning}}]
    {erp}{ERP}{Enterprise Resource Planning}

\newacronym[description={\glslink{htmlg}{Hypertext Markup Language}}]
    {html}{HTML}{Hypertext Markup Language}

\newacronym[description={\glslink{cssg}{Cascading Style Sheets}}]
    {css}{CSS}{Cascading Style Sheets}

\newacronym[description={\glslink{jsg}{JavaScript}}]
    {js}{JS}{JavaScript}

\newacronym[description={\glslink{iclg}{In-Context Learning}}]
    {icl}{ICL}{In-Context Learning}

\newacronym[description={\glslink{fslg}{Few-Shot Learning}}]
    {fsl}{FSL}{Few-Shot Learning}

\newacronym[description={\glslink{vifg}{Visual Information Flow}}]
    {vif}{VIF}{Visual Information Flow}

\newacronym[description={\glslink{mbtig}{Myers-Briggs Type Indicator}}]
    {mbti}{MBTI}{Myers-Briggs Type Indicator}
    
% Glossary entries
\newglossaryentry{datavizg} {
    name=Data visualization,
    text=Data Visualization,
    sort=datavisualization,
    description={È il processo di rappresentazione visiva dei dati attraverso grafici, tabelle e altre forme di visualizzazione grafica.
    Viene utilizzato per facilitare l'interpretazione e l'analisi delle informazioni, permettendo di identificare tendenze, \emph{pattern} e \emph{insight} 
    in modo chiaro e intuitivo, migliorandone la comprensione.}
}

\newglossaryentry{d3g} {
    name=D3.js,
    text=D3.js,
    sort=d3,
    description={Libreria JavaScript \emph{open source} usata per la creazione, a partire da dati organizzati, di visualizzazioni interattive che siano visibili attraverso un comune browser.
    Il suo nome, infatti, sta per \emph{Data-Driven Documents} (documenti basati sui dati). 
    In particolare, consente di creare grafici interattivi e dinamici sul web, utilizzando HTML, SVG e CSS}
}

\newglossaryentry{ictg} {
    name=\glslink{ict}{ICT},
    text=Information and Communication Technology,
    sort=ict,
    description={descrizione}
}

\newglossaryentry{erpg} {
    name=\glslink{erp}{ERP},
    text=Enterprise Resource Planning,
    sort=erp,
    description={descrizione}
}

\newglossaryentry{ebusinessg} {
    name=e-business,
    text=e-business,
    sort=ebusiness,
    description={descrizione}
}

\newglossaryentry{businessintg} {
    name=business intelligence,
    text=business intelligence,
    sort=businessintelligence,
    description={descrizione}
}

\newglossaryentry{serverfarmg} {
    name=server farm,
    text=server farm,
    sort=serverfarm,
    description={descrizione}
}

\newglossaryentry{mlg} {
    name=Machine Learning,
    text=Machine Learning,
    sort=machinelearning,
    description={descrizione}
}

\newglossaryentry{htmlg} {
    name=\glslink{html}{HTML},
    text=Hypertext Markup Language,
    sort=html,
    description={descrizione}
}

\newglossaryentry{cssg} {
    name=\glslink{css}{CSS},
    text=Cascading Style Sheets,
    sort=css,
    description={descrizione}
}

\newglossaryentry{jsg} {
    name=\glslink{js}{JS},
    text=JavaScript,
    sort=js,
    description={descrizione}
}

\newglossaryentry{llmg} {
    name=\glslink{llm}{LLM},
    text=Large Language Model,
    sort=llm,
    description={Tipologia di modello di AI progettato per comprendere e
    generare testo con una capacità simile a quella umana; utilizzando reti neurali
    profonde, questi modelli apprendono regole linguistiche, coerenza del contesto e
    stili di scrittura da grandi quantità di dati testuali}
}

\newglossaryentry{infoloadg} {
    name=Information Overload,
    text=Information Overload,
    sort=informationoverload,
    description={descrizione}
}

\newglossaryentry{bigdatag} {
    name=Big Data,
    text=Big Data,
    sort=bigdata,
    description={descrizione}
}

\newglossaryentry{datascienceg} {
    name=Data Science,
    text=Data Science,
    sort=datascience,
    description={descrizione}
}

\newglossaryentry{chartjunkg} {
    name=chartjunk,
    text=chartjunk,
    sort=chartjunk,
    description={descrizione}
}

\newglossaryentry{sistemadiregoleg} {
    name=sistema di regole,
    text=sistema di regole,
    sort=sistemadiregole,
    description={descrizione}
}

\newglossaryentry{promptengg} {
    name=Prompt Engineering,
    text=Prompt Engineering,
    sort=promptengineering,
    description={descrizione}
}

\newglossaryentry{fslg} {
    name=\glslink{fsl}{FSL},
    text=Few-Shot Learning,
    sort=fsl,
    description={descrizione}
}

\newglossaryentry{iclg} {
    name=\glslink{icl}{ICL},
    text=In-Context Learning,
    sort=icl,
    description={descrizione}
}

\newglossaryentry{llama7bg} {
    name=llama 7b,
    text=llama 7b,
    sort=llama7b,
    description={descrizione}
}

\newglossaryentry{llamacppg} {
    name=llama.cpp,
    text=llama.cpp,
    sort=llamacpp,
    description={descrizione}
}

\newglossaryentry{motoreregoleg} {
    name=motore di regole,
    text=motore di regole,
    sort=motorediregole,
    description={descrizione}
}

\newglossaryentry{jsonrulesg} {
    name=json-rules-engine,
    text=json-rules-engine,
    sort=jsonrulesengine,
    description={descrizione}
}

\newglossaryentry{jsong} {
    name=JSON,
    text=JSON,
    sort=json,
    description={descrizione}
}

\newglossaryentry{expdataanalysisg} {
    name=Exploratory Data Analysis,
    text=Exploratory Data Analysis,
    sort=exploratorydataanalysis,
    description={descrizione}
}

\newglossaryentry{grammaticag} {
    name=grammatica,
    text=grammatica,
    sort=grammatica,
    description={descrizione}
}

\newglossaryentry{vifg} {
    name=\glslink{vif}{VIF},
    text=Visual Information Flow,
    sort=vif,
    description={descrizione}
}


\newglossaryentry{mbtig} {
    name=\glslink{mbti}{MBTI},
    text=Myers-Briggs Type Indicator,
    sort=myersbriggstypeindicator,
    description={descrizione}
}
