\chapter{Appendice A}\label{cap:appendix}

\section{Grafici risultanti da Chart-chooser}
Di seguito si riporta una tabella contenente tutti i grafici possibili individuati, insieme ad un breve consiglio 
di utilizzo e alle \emph{regole} del \gls{sistemadiregoleg} di \emph{Chart-chooser} che possono portare all'uso 
di tale grafico. Per quanto riguarda la denominazione delle regole si faccia riferimento alle figure \ref{fig:rules_pt1} 
e \ref{fig:rules_pt2}.

Si precisa che i grafici identificati sono frutto dell'analisi di diverse risorse, principalmente \cite{site:data-to-viz} e \cite{site:vis_vocabulary}.
\begin{xltabular}{\columnwidth}{|p{0.15\columnwidth}|X|p{0.27\columnwidth}|}
    \hline
    \rowcolor{gray!20}
    \textbf{Nome} & \textbf{Regole} & \textbf{Consiglio d'uso} \\
    \endhead
    \hline
    Diverging Bar & 
    \vspace{-3.5mm}
    \begin{itemize}[noitemsep,topsep=0pt, left=0pt]
        \item deviationNumUniv
    \end{itemize} & 
    Quando la precisione dei valori è importante, quando i valori sono discreti, quando il numero dei valori non è troppo elevato. \\
    \hline
    Diverging Area Chart & 
    \vspace{-3.5mm}
    \begin{itemize}[noitemsep,topsep=0pt, left=0pt]
        \item deviationNumUniv
    \end{itemize} & 
    Quando si vuole evidenziare un trend, con valori continui o serie temporali. \\
    \hline
    Diverging Stacked Bar & 
    \vspace{-3.5mm}
    \begin{itemize}[noitemsep,topsep=0pt, left=0pt]
        \item deviationNumMultiOrd
        \item deviationMixMultiNumOneObs
    \end{itemize} & 
    Quando si vuole evidenziare la divergenza per componenti di un gruppo, quando il numero delle variabili non è elevato. \\
    \hline
    Butterfly Chart & 
    \vspace{-3.5mm}
    \begin{itemize}[noitemsep,topsep=0pt, left=0pt]
        \item deviationNumMultiOrd
    \end{itemize} & 
    Quando si hanno due variabili numeriche rappresentanti gruppi contrastanti. \\
    \hline
    Connected Scatterplot & 
    \vspace{-3.5mm}
    \begin{itemize}[noitemsep,topsep=0pt, left=0pt]
        \item correlationNumMultiOrd
        \item correlationMixMultiNumOrd
        \item timeNumMultiOrd
        \item timeMixMultiNumOrd
    \end{itemize} & 
    Quando l'ordine dei punti è importante, quando si vogliono mostrare come si evolvono le relazioni tra un punto e l'altro. \\
    \hline
    Scatterplot & 
    \vspace{-3.5mm}
    \begin{itemize}[noitemsep,topsep=0pt, left=0pt]
        \item correlationNumTwoVarUnord
    \end{itemize} & 
    Quando si vogliono identificare anomalie, si vuole visualizzare la distribuzione dei punti. \\
    \hline
    Correlogram & 
    \vspace{-3.5mm}
    \begin{itemize}[noitemsep,topsep=0pt, left=0pt]
        \item correlationNumLotsVarUnord
        \item correlationMixMultiNumUnord
    \end{itemize} & 
    Quando si vuole esaminare correlazioni a coppie tra molte variabili. \\
    \hline
    Bubble Chart & 
    \vspace{-3.5mm}
    \begin{itemize}[noitemsep,topsep=0pt, left=0pt]
        \item correlationNumThreeVarUnord
        \item rankingNumThreeVarUnord
        \item magnitudeNumThreeVarUnord
    \end{itemize} & 
    Quando si vuole mostrare la correlazione tra tre variabili simultaneamente. \\
    \hline
    Line+Column Chart & 
    \vspace{-3.5mm}
    \begin{itemize}[noitemsep,topsep=0pt, left=0pt]
        \item correlationNumLotsVarUnord
        \item timeNumMultiOrd
    \end{itemize} & 
    Quando si vuole mostrare evoluzioni della relazioni tra dati con due diverse unità di misura. \\
    \hline
    Ordered Bar & 
    \vspace{-3.5mm}
    \begin{itemize}[noitemsep,topsep=0pt, left=0pt]
        \item rankingCatUniv
        \item rankingMixUnivOneObs
    \end{itemize} & 
    Quando si hanno tante categorie. \\
    \hline
    Ordered Column & 
    \vspace{-3.5mm}
    \begin{itemize}[noitemsep,topsep=0pt, left=0pt]
        \item rankingCatUniv
        \item rankingMixUnivOneObs
    \end{itemize} & 
    Quando si hanno poche categorie. \\
    \hline
    Lollipop Chart & 
    \vspace{-3.5mm}
    \begin{itemize}[noitemsep,topsep=0pt, left=0pt]
        \item rankingCatUniv
        \item rankingMixUnivOneObs
        \item timeNumMultiOrd
        \item magnitudeCatUniv
        \item magnitudeMixUnivOneObs
    \end{itemize} & 
    Da usare come alternativa all'\emph{Ordered Bar} o all'\emph{Ordered Column}. \\
    \hline
    Slope Chart & 
    \vspace{-3.5mm}
    \begin{itemize}[noitemsep,topsep=0pt, left=0pt]
        \item rankingCatMultiInd
        \item rankingMixMultiNumOneObs
        \item timeCatMultiInd
        \item timeMixMultiNumOneObs
    \end{itemize} & 
    Quando si vogliono visualizzare dei cambiamenti discreti nel tempo, quando si hanno poche categorie. \\
    \hline
    Dot Strip Plot & 
    \vspace{-3.5mm}
    \begin{itemize}[noitemsep,topsep=0pt, left=0pt]
        \item rankingMixUnivMoreObs
        \item distributionNumLotsVarUnord
    \end{itemize} & 
    Quando si vogliono visualizzare ranking e distribuzione di dataset ampi. \\
    \hline
    Histogram & 
    \vspace{-3.5mm}
    \begin{itemize}[noitemsep,topsep=0pt, left=0pt]
        \item distributionNumUniv
        \item distributionNumTwoVarUnord
        \item distributionMixUnivMoreObs
        \item distributionMixUnivOneObs
    \end{itemize} & 
    Quando si vuole visualizzare frequenze su intervalli, quando si hanno pochi punti, quando si vuole una rappresentazione semplice adatta a tutti. \\
    \hline
    Density Plot & 
    \vspace{-3.5mm}
    \begin{itemize}[noitemsep,topsep=0pt, left=0pt]
        \item distributionMixUnivMoreObs
        \item distributionNumUniv
    \end{itemize} & 
    Quando si ha una distribuzione non categorizzata in bin, quando si vuole mostrare la ``forma'' dei dati. \\
    \hline
    Cumulative curve & 
    \vspace{-3.5mm}
    \begin{itemize}[noitemsep,topsep=0pt, left=0pt]
        \item distributionNumUniv
        \item distributionMixUnivMoreObs
    \end{itemize} & 
    Quando si vuole mostrare le proporzioni dei dati e le loro tendenze. \\
    \hline
    Boxplot & 
    \vspace{-3.5mm}
    \begin{itemize}[noitemsep,topsep=0pt, left=0pt]
        \item distributionNumLotsVarUnord
        \item distributionMixUnivMoreObs
        \item distributionMixMultiNumUnord
        \item distributionMixMultiCatNestMoreObs
        \item distributionMixMultiCatSubMoreObs
    \end{itemize} & 
    Quando si hanno pochi punti, quando si vogliono mostrare statistiche come mediana, quartili e casi anomali. \\
    \hline
    Violin Plot & 
    \vspace{-3.5mm}
    \begin{itemize}[noitemsep,topsep=0pt, left=0pt]
        \item distributionNumLotsVarUnord
        \item distributionMixUnivMoreObs
        \item distributionMixMultiNumUnord
        \item distributionMixMultiCatNestMoreObs
        \item distributionMixMultiCatSubMoreObs
    \end{itemize} & 
    Quando si hanno tanti punti, quando si vuole combinare \emph{Boxplot} con \emph{Density Plot}. \\
    \hline
    2D Density Plot & 
    \vspace{-3.5mm}
    \begin{itemize}[noitemsep,topsep=0pt, left=0pt]
        \item distributionNumTwoVarUnord
        \item distributionMixMultiNumUnord
    \end{itemize} & 
    Quando si hanno tanti punti, quando si ha un dataset ampio. \\
    \hline
    Ridgeline & 
    \vspace{-3.5mm}
    \begin{itemize}[noitemsep,topsep=0pt, left=0pt]
        \item distributionNumLotsVarUnord
        \item distributionMixUnivMoreObs
    \end{itemize} & 
    Quando si hanno tanti punti, quando si vogliono visualizzare più distribuzioni. \\
    \hline
    Calendar Heatmap & 
    \vspace{-3.5mm}
    \begin{itemize}[noitemsep,topsep=0pt, left=0pt]
        \item timeNumUniv
    \end{itemize} & 
    Quando si hanno dati giornalieri e si vogliono visualizzare pattern che si ripetono. \\
    \hline
    Seismogram & 
    \vspace{-3.5mm}
    \begin{itemize}[noitemsep,topsep=0pt, left=0pt]
        \item timeNumUniv
    \end{itemize} & 
    Quando si vogliono mostrare oscillazioni o cambiamenti repentini o alti nel tempo. \\
    \hline
    Line Chart & 
    \vspace{-3.5mm}
    \begin{itemize}[noitemsep,topsep=0pt, left=0pt]
        \item timeNumMultiOrd
        \item timeMixMultiNumOrd
    \end{itemize} & 
    Quando si vogliono mostrare dei trend continui nel tempo. \\
    \hline
    Area Chart & 
    \vspace{-3.5mm}
    \begin{itemize}[noitemsep,topsep=0pt, left=0pt]
        \item timeNumMultiOrd
    \end{itemize} & 
    Quando si vogliono mostrare cambiamenti cumulativi nel tempo. \\
    \hline
    Fan Chart & 
    \vspace{-3.5mm}
    \begin{itemize}[noitemsep,topsep=0pt, left=0pt]
        \item timeNumMultiOrd
        \item timeMixMultiNumOrd
    \end{itemize} & 
    Quando si hanno dati predittivi e si vuole rappresentare incertezza o un range di possibili risultati. \\
    \hline
    Priestley Timeline & 
    \vspace{-3.5mm}
    \begin{itemize}[noitemsep,topsep=0pt, left=0pt]
        \item timeCatUniv
    \end{itemize} & 
    Quando si vuole mostrare una sequenza di eventi nel tempo. \\
    \hline
    Circle Timeline & 
    \vspace{-3.5mm}
    \begin{itemize}[noitemsep,topsep=0pt, left=0pt]
        \item timeMixUnivMoreObs
    \end{itemize} & 
    Quando si vuole mostrare la grandezza di valori discreti che cambiano nel tempo. \\
    \hline
    Stacked Area Chart & 
    \vspace{-3.5mm}
    \begin{itemize}[noitemsep,topsep=0pt, left=0pt]
        \item timeMixMultiNumOrd
    \end{itemize} & 
    Quando si vogliono mostrare cambiamenti cumulativi nel tempo enfatizzandole la grandezza, quando si hanno poche categorie. \\
    \hline
    Stream Graph & 
    \vspace{-3.5mm}
    \begin{itemize}[noitemsep,topsep=0pt, left=0pt]
        \item timeMixMultiNumOrd
    \end{itemize} & 
    Alternativa a \emph{Stacked Area Chart} che enfatizza il flusso di cambiamenti cumulativi nel tempo, quando si hanno poche categorie. \\
    \hline
    Pie Chart & 
    \vspace{-3.5mm}
    \begin{itemize}[noitemsep,topsep=0pt, left=0pt]
        \item compositionCatUniv
        \item compositionMixUnivOneObs
    \end{itemize} & 
    Quando si ha un dataset piccolo, quando si vogliono evidenziare le proporzioni. \\
    \hline
    Donut Chart & 
    \vspace{-3.5mm}
    \begin{itemize}[noitemsep,topsep=0pt, left=0pt]
        \item compositionCatUniv
        \item compositionMixUnivOneObs
    \end{itemize} & 
    Alternativa a \emph{Pie Chart}, quando si ha un dataset piccolo, quando si vogliono evidenziare le proporzioni. \\
    \hline
    Waterfall & 
    \vspace{-3.5mm}
    \begin{itemize}[noitemsep,topsep=0pt, left=0pt]
        \item compositionCatUniv
        \item compositionMixUnivOneObs
        \item flowCatUniv
    \end{itemize} & 
    Quando si vogliono mostrare come i valori influiscono nel totale. \\
    \hline
    Waffle Chart & 
    \vspace{-3.5mm}
    \begin{itemize}[noitemsep,topsep=0pt, left=0pt]
        \item compositionCatUniv
        \item compositionMixUnivOneObs
    \end{itemize} & 
    Quando si vuole mostrare la composizione di una categoria in griglia. \\
    \hline
    Venn Diagram & 
    \vspace{-3.5mm}
    \begin{itemize}[noitemsep,topsep=0pt, left=0pt]
        \item compositionCatMultiInd
    \end{itemize} & 
    Quando si vogliono mostrare le parti che si sovrappongono tra diverse categorie. \\
    \hline
    Treemap & 
    \vspace{-3.5mm}
    \begin{itemize}[noitemsep,topsep=0pt, left=0pt]
        \item compositionCatMultiNest
        \item compositionMixMultiCatNestOneObs
    \end{itemize} & 
    Quando si vogliono mostrare dati gerarchici, quando si vogliono mostrare tanti valori in uno spazio compatto. \\
    \hline
    Sunburst & 
    \vspace{-3.5mm}
    \begin{itemize}[noitemsep,topsep=0pt, left=0pt]
        \item compositionCatMultiNest
        \item compositionMixMultiCatNestOneObs
    \end{itemize} & 
    Quando si vogliono mostrare dati gerarchici. \\
    \hline
    Icicle & 
    \vspace{-3.5mm}
    \begin{itemize}[noitemsep,topsep=0pt, left=0pt]
        \item compositionCatMultiNest
        \item compositionMixMultiCatNestOneObs
    \end{itemize} & 
    Alternativa al Sunburst, quando si vogliono mostrare dati gerarchici con settori rettangolari. \\
    \hline
    Stacked Column Chart & 
    \vspace{-3.5mm}
    \begin{itemize}[noitemsep,topsep=0pt, left=0pt]
        \item compositionCatMultiSub
        \item compositionMixMultiCatSubOneObs
    \end{itemize} & 
    Quando si vuole mostrare la composizione di diverse categorie o su periodi di tempo. \\
    \hline
    Stacked Bar Chart & 
    \vspace{-3.5mm}
    \begin{itemize}[noitemsep,topsep=0pt, left=0pt]
        \item compositionCatMultiSub
        \item compositionMixMultiCatSubOneObs
        \item magnitudeMixMultiCatSubOneObs
    \end{itemize} & 
    Quando si vuole mostrare la composizione di diverse categorie. \\
    \hline
    Proportional Stacked Bar & 
    \vspace{-3.5mm}
    \begin{itemize}[noitemsep,topsep=0pt, left=0pt]
        \item compositionCatMultiSub
        \item compositionMixUnivOneObs
        \item compositionCatMultiSub
        \item magnitudeMixUnivOneObs
    \end{itemize} & 
    Quando si vuole mostrare la contribuzione relativa delle categorie sul totale. \\
    \hline
    Dendogram & 
    \vspace{-3.5mm}
    \begin{itemize}[noitemsep,topsep=0pt, left=0pt]
        \item compositionCatMultiNest
        \item compositionMixMultiCatNestOneObs
    \end{itemize} & 
    Quando si vuole rappresentare una composizione gerarchica. \\
    \hline
    Circular Packing & 
    \vspace{-3.5mm}
    \begin{itemize}[noitemsep,topsep=0pt, left=0pt]
        \item compositionCatMultiNest
        \item compositionMixMultiCatNestOneObs
    \end{itemize} & 
    Quando si vuole rappresentare una composizione gerarchica enfatizzando la grandezza della categoria. \\
    \hline
    Pictogram & 
    \vspace{-3.5mm}
    \begin{itemize}[noitemsep,topsep=0pt, left=0pt]
        \item magnitudeCatUniv
        \item magnitudeMixUnivOneObs
    \end{itemize} & 
    Quando si vuole rappresentare quantità con simboli. \\
    \hline
    Grouped Barplot & 
    \vspace{-3.5mm}
    \begin{itemize}[noitemsep,topsep=0pt, left=0pt]
        \item compositionCatMultiSub
        \item magnitudeMixMultiNumOneObs
        \item magnitudeMixMultiCatSubOneObs
    \end{itemize} & 
    Quando si vuole comparare la grandezza di gruppi tra loro collegati, quando si hanno pochi gruppi. \\
    \hline
    Radar Chart & 
    \vspace{-3.5mm}
    \begin{itemize}[noitemsep,topsep=0pt, left=0pt]
        \item magnitudeMixMultiNumOneObs
    \end{itemize} & 
    Quando si vuole comparare multiple variabili per diverse categorie. \\
    \hline
    Dot Density & 
    \vspace{-3.5mm}
    \begin{itemize}[noitemsep,topsep=0pt, left=0pt]
        \item spatialNumTwoVarUnord
    \end{itemize} & 
    Quando si vogliono visualizzare punti su una mappa, quando si vuole mostrare la densità in diverse regioni. \\
    \hline
    Proportional Symbol & 
    \vspace{-3.5mm}
    \begin{itemize}[noitemsep,topsep=0pt, left=0pt]
        \item spatialNumThreeVarUnord
    \end{itemize} & 
    Quando si vuole mostrare la magnitudine del dato su una mappa. \\
    \hline
    Flow Map & 
    \vspace{-3.5mm}
    \begin{itemize}[noitemsep,topsep=0pt, left=0pt]
        \item spatialNumFourFiveVarUnord
        \item spatialCatTwoVarInd
        \item spatialMixTwoCatInd
    \end{itemize} & 
    Quando si vogliono visualizzare un flusso tra diversi punti sulla mappa. \\
    \hline
    Basic Choropleth & 
    \vspace{-3.5mm}
    \begin{itemize}[noitemsep,topsep=0pt, left=0pt]
        \item spatialCatUniv
        \item spatialMixUnivOneObs
    \end{itemize} & 
    Quando si vuole mostrare la distribuzione di una variabile tra diverse regioni geografiche. \\
    \hline
    Equalized Cartogram & 
    \vspace{-3.5mm}
    \begin{itemize}[noitemsep,topsep=0pt, left=0pt]
        \item spatialCatUniv
        \item spatialMixUnivOneObs
    \end{itemize} & 
    Quando si vuole mostrare la distribuzione di una variabile tra diverse regioni geografiche in maniera uniforme. \\
    \hline
    Contour Map & 
    \vspace{-3.5mm}
    \begin{itemize}[noitemsep,topsep=0pt, left=0pt]
        \item spatialCatUniv
        \item spatialMixUnivOneObs
    \end{itemize} & 
    Quando si vuole mostrare la divergenza di variabili continue su un'area geografica. \\
    \hline
    Scaled Cartogram & 
    \vspace{-3.5mm}
    \begin{itemize}[noitemsep,topsep=0pt, left=0pt]
        \item spatialCatUniv
        \item spatialMixUnivOneObs
    \end{itemize} & 
    Quando si vuole rappresentare la magnitudine di una variabile su una mappa. \\
    \hline
    Heatmap & 
    \vspace{-3.5mm}
    \begin{itemize}[noitemsep,topsep=0pt, left=0pt]
        \item flowCatMultiInd
        \item flowCatMultiSub
        \item flowMixMultiCatSubOneObs
        \item flowMixMultiCatInd
    \end{itemize} & 
    Quando si vuole visualizzare l'intensità di flussi in forma tabulare. \\
    \hline
    Sankey Diagram & 
    \vspace{-3.5mm}
    \begin{itemize}[noitemsep,topsep=0pt, left=0pt]
        \item flowCatMultiInd
        \item flowCatMultiSub
        \item flowMixMultiNumOneOb
        \item flowMixMultiCatSubOneObs
        \item flowMixMultiCatInd
    \end{itemize} & 
    Quando si vuole visualizzare la struttura del flusso, quando si vuole mostrare la distribuzione proporzionale del flusso. \\
    \hline
    Network Diagram & 
    \vspace{-3.5mm}
    \begin{itemize}[noitemsep,topsep=0pt, left=0pt]
        \item flowCatMultiInd
        \item flowMixMultiCatInd
    \end{itemize} & 
    Quando si vogliono mostrare connessioni tra più entità enfatizzando comunità e intensità dei flussi. \\
    \hline
    Chord Diagram & 
    \vspace{-3.5mm}
    \begin{itemize}[noitemsep,topsep=0pt, left=0pt]
        \item flowCatMultiInd
        \item flowMixMultiCatInd
    \end{itemize} & 
    Quando si vogliono visualizzare connessioni tra più entità enfatizzando l'intensità tra le connessioni. \\
    \hline
    Arc Diagram & 
    \vspace{-3.5mm}
    \begin{itemize}[noitemsep,topsep=0pt, left=0pt]
        \item flowCatMultiInd
        \item flowMixMultiCatInd
    \end{itemize} & 
    Quando si vogliono visualizzare connessioni tra più entità enfatizzando l'intensità tra le connessioni. \\
    \hline
    \caption{Grafici risultanti da Chart-chooser}
    \label{tab:grafici}
\end{xltabular}